\documentclass[xcolor=dvipsnames, notes]{beamer}

%%%%%%%%%%%%%%%%%%%
% Packages/Macros %
%%%%%%%%%%%%%%%%%%%
\usepackage{amsmath,amsfonts,amssymb,amsthm, mathtools, mathrsfs, faktor, bbm}
\usepackage{wrapfig}
\usepackage{array}
\usepackage{graphicx}
\usepackage{tikz}
\usepackage{tikz-cd}
\usepackage{enumerate}

\usepackage{import}
\usepackage{xifthen}
\usepackage{pdfpages}
\usepackage{transparent}
\newcommand{\incfig}[1]{%
  \def\svgwidth{0.5\columnwidth}
  \import{./images/}{#1.pdf_tex}
}
\newcommand*\circled[1]{\tikz[baseline=(char.base)]{
            \node[shape=circle,draw,inner sep=2pt] (char) {#1};}}


\newcommand{\im}{\mathrm{Im}}
\newcommand{\re}{\mathrm{Re}}
\newcommand{\res}{\mathrm{Res}}
\newcommand{\pv}{\mathrm{p.v.}}
\DeclareMathOperator{\tr}{tr}
\DeclareMathOperator{\Log}{Log}
\DeclareMathOperator{\adj}{adj}
\DeclareMathOperator{\Hom}{Hom}
\DeclareMathOperator{\rk}{rk}
\DeclareMathOperator{\divv}{div}

% General shortcuts
\newcommand{\RR}{\mathbb{R}}
\newcommand{\CC}{\mathbb{C}}
\newcommand{\ZZ}{\mathbb{Z}}
\newcommand{\QQ}{\mathbb{Q}}
\newcommand{\NN}{\mathbb{N}}
\newcommand{\HH}{\mathbb{H}}
\newcommand{\MM}{\mathcal{M}}
\newcommand{\Id}{\mathbbm{1}}
\newcommand{\wms}{\textsc{wms} }
\newcommand{\wma}{\textsc{wma} }
\newcommand{\wwlog}{\textsc{wlog} }
\newcommand{\Arg}{\text{Arg}}
\renewcommand\qedsymbol{$\blacksquare$}


%\usepackage[utf8]{inputenc}
\usepackage{pgfpages}

\usepackage{amssymb,latexsym,amsmath,relsize,multicol, amsthm, tikz,tikz-cd, wrapfig, setspace,framed,xcolor,array , array, faktor, listings, mathrsfs , blkarray,booktabs,bigstrut}

\usepackage{soul}
\renewcommand<>{\hl}[1]{\only#2{\beameroriginal{\hl}}{#1}}

% https://tex.stackexchange.com/questions/41683/why-is-it-that-coloring-in-soul-in-beamer-is-not-visible
\makeatletter
\newcommand\SoulColor{%
  \let\set@color\beamerorig@set@color
  \let\reset@color\beamerorig@reset@color}
\makeatother
\SoulColor

\usetikzlibrary{arrows}

\usepackage[framemethod=TikZ]{mdframed}
\mdfsetup{nobreak=true}

\usepackage{stackengine}
\newcommand\xrowht[2][0]{\addstackgap[.5\dimexpr#2\relax]{\vphantom{#1}}}

\usepackage{hhline}

% \usepackage{tikzlings}

%%%%%%%%%%%%%%%%%%
% Beamer Options %
%%%%%%%%%%%%%%%%%%

%\usetheme{Frankfurt}
\usetheme{Antibes}

% Color definitions:
\definecolor{mypurple}{RGB}{179,120,211}
\definecolor{darkpurple}{RGB}{130,98,168}
\definecolor{mygrey}{RGB}{164,167,172}
\definecolor{mygrey2}{RGB}{217,219,220}
\definecolor{fgreen}{RGB}{34,169,75}
\definecolor{gpurple}{RGB}{152,68,158}

%\useoutertheme{infolines} % Alternatively: miniframes, infolines, split
\useinnertheme{circles}
\definecolor{goodgood}{RGB}{127, 23, 52} % UBC Blue (primary)
\definecolor{goodgood1}{RGB}{46, 82, 102} % UBC Blue (primary)
\definecolor{goodgood2}{RGB}{203, 133, 137}
\usecolortheme[named=goodgood]{structure}
%\usecolortheme[named=Mahogany]{structure} % Sample dvipsnames color

%\setbeamercolor{structure}{fg=gpurple} % Title box and slide title box color
%\setbeamercolor{frametitle}{fg=white} % slide title text color
%\setbeamercolor{section in head/foot}{bg=gpurple} % frame background color
%\setbeamercolor{section in head/foot}{fg=white} % frame text color

\setbeamertemplate{navigation symbols}{} % disable navigation icons
\setbeamertemplate{items}[circle]

% Beamer theme font
\usefonttheme{serif}

% % --- page number ---
% \setbeamertemplate{footline}{%
% 	\raisebox{10pt}{\makebox[\paperwidth]{\hfill\makebox[7em]{\normalsize\texttt{\insertframenumber/\inserttotalframenumber}}}}%
% }

% Presenter's note
\setbeameroption{show notes on second screen}
%%%%%%%%%%%%%%%%%%%%%%%%%%%%%%
% Theorem/Proof Environments %
%%%%%%%%%%%%%%%%%%%%%%%%%%%%%%

\renewenvironment{definition}[1][]{\par\medskip\noindent \textbf{Definition:} \textit{#1} \rmfamily \\}{\vspace{4pt}}

\renewenvironment{example}[1][]{\par\medskip\noindent \textbf{Example.} \rmfamily}{\medskip}

\newenvironment{proposition}[1][]{\par\medskip\noindent \textbf{Proposition:} \rmfamily \\}{ \vspace{4pt}}

\newenvironment{mtheorem}[1][]{\begin{mdframed}[roundcorner=10pt] \refstepcounter{theorem}\par\medskip\noindent \textbf{Theorem~\thetheorem.} [#1] \rmfamily \\}{ \vspace{4pt}\end{mdframed}\vspace{-5pt}}
\numberwithin{theorem}{section}

\newenvironment{iproposition}[1][]{\begin{mdframed}[roundcorner=10pt] \refstepcounter{theorem}\par\medskip\noindent \textbf{Proposition~\thetheorem.} [#1] \rmfamily \\}{ \vspace{4pt}\end{mdframed}\vspace{-5pt}}
\numberwithin{theorem}{section}

\newenvironment{conjecture}[1][]{\begin{mdframed}[roundcorner=10pt] \refstepcounter{theorem}\par\medskip\noindent \textbf{Conjecture~\thetheorem.} \rmfamily \\}{ \vspace{4pt}\end{mdframed}\vspace{-5pt}}
\numberwithin{theorem}{section}

\newenvironment{nconjecture}[1][]{\begin{mdframed}[roundcorner=10pt] \refstepcounter{theorem}\par\medskip\noindent \textbf{Conjecture~\thetheorem.(#1)} \rmfamily \\}{ \vspace{4pt}\end{mdframed}\vspace{-5pt}}
\numberwithin{theorem}{section}

\theoremstyle{remark}
\newtheorem*{remark}{Remark}

%%%%%%%%%%%%%%%%%%%
% Custom Commands %
%%%%%%%%%%%%%%%%%%%

\newcommand{\paren}[1]{\left( #1 \right)}
\newcommand{\bracket}[1]{\left[ #1 \right]}
\newcommand{\cparen}[1]{\left\{ #1 \right\}}
\newcommand{\eval}[3]{\left. #1 \right|_{#2}^{#3}}
\newcommand{\abs}[1]{\left| #1 \right|}


%%%%%%%%%%%%
% Document %
%%%%%%%%%%%%

\title{Searching for Holes in the Matrix Universe}
\author{Lucas Kerbs}
\date{Spring 2022}

\begin{document}
\begin{frame}[plain]
  \maketitle
  \note[item]{ Eventual goal: lift the tools of algebraic topology to spaces of
    matrices}
  \note[item]{ If we have a fixed size, use classical results but this
    doesn't work for multiple sizes.}
  \note[item]{ For multiple (which hopefully I can convince you is interesting)
    we need to develop some hefty tools}
  \note[item]{ To do so, we need to go back to our mathematical roots}
\end{frame}

\section{Part I: Objects and Maps}

\begin{frame}
  \begin{center}
    \textcolor{goodgood}{\huge{Objects and Maps}} \\
    \textcolor{goodgood2}{\large{A Naive Attempt}} \\
  \end{center}
  \note[item]{ That's right---objects and maps.}
  \note[item]{Our Naive attempt involves that looking at lifting functions on
    \(\RR \) or \(\CC \) to accept matrices as their input.}
\end{frame}

\subsection{Functional Calculus}

\begin{frame}{Functional Calculus}
  \onslide<1->{Let \(f \in \RR [x]\) and \(A \in M_k(\CC)\) be self adjoint.}\\
  \onslide<2->{\(A\) is diagonalizable as \(A = U \Lambda U^{*}\)}
  \begin{align*}
    \onslide<3->{f(A) &= a_nA^n + \cdots + a_1A + a_0 I_k  \\}
    \onslide<4->{&= a_n \left( U\Lambda U^* \right) ^n + \cdots + a_1 U\Lambda U^* + a_0 I_k  \\}
    \onslide<5->{&= a_n U\Lambda^n U^* + \cdots + a_1 U\Lambda U^* + a_0 I_k  \\}
    \onslide<6->{&= U \left( a_n\Lambda ^n + \cdots + a_1\Lambda + a_0 I_k \right) U^*  \\}
    \onslide<7->{&= U \left( f(\Lambda) \right) U^*}
  \end{align*}
  \[
   \onslide<8->{f \left( \begin{bmatrix} \lambda_1 &  &  \\  & \ddots &  \\  &  & \lambda_n \end{bmatrix}  \right)}
   \onslide<9->{ =\begin{bmatrix} f(\lambda_1) &  &  \\  & \ddots &  \\  &  & f(\lambda_n) \end{bmatrix}}
  \]
  \note{
    \begin{itemize}
    \item The naive attempt is modeled after the behavior of polynomials. We could
    plug in an arbitrary matrix into a polynomial without too much trouble, but
    requiring SA gives us some special behavior.
    \item Since we are SA, we can diagonalize
    \item Watch what happens when we plug this into our polynomial
    \end{itemize}
  }
\end{frame}

\begin{frame}
  \onslide<+->{
  Let \(\HH_n\) be the set of \(n\times n\) self adjoint matrices, and define
  \[
    \HH = \bigcup_{n \in \NN } \HH_n, \qquad \MM = \bigcup_{n \in \NN }M_n(\CC)
  \]}
  \onslide<+->{
  \begin{definition}
    Let \(g:[a,b]\to \CC \) and \(D \subset\HH \) denote the set of self adjoint
    matrices with their spectrum in \([a,b]\).}
    \onslide<+->{Then
    \begin{align*}
      g: D &\longrightarrow \MM  \\
         X=U\Lambda U^{*}  &\longmapsto U
                     \begin{bmatrix} g(\lambda_1) & &\\ &\ddots& \\ & & g(\lambda_n) \end{bmatrix}
                    U^*
    .\end{align*}}
  \end{definition}
  \note{
    \begin{itemize}
      \item With the polynomial case in mind, we can extend a general function.
            First, a piece of notation
      \item Lets grab a function on the real line and the self adjoint
            matrices with their spectrum in that domain
      \item Then we can lift \(g\) by emulating the behavior of polynomials.
      \item The natural question now is ``what can we do with these functions''
    \end{itemize}
  }
\end{frame}

\begin{frame}{Directional Derivative}
  \onslide<+->{
  \begin{Definition}[Directional Derivative]
    Fix some \(X \in \HH_n\). The derivative of \(f\) at \(X\) in the direction
    \(H \in M_n(\CC)\) is
    \[
      Df(X)[H] = \lim_{t \to 0} \frac{f(X+tH)-f(X)}{t}
    \]}
  \onslide<+->{
    Alternatively,
    \[
    Df(X)[H] = \left.\frac{df(X+tH)}{dt}\right|_{t=0}
    \]}
  \end{Definition}
  \note{
    \begin{itemize}
      \item We can define a directional derivative---as long as we are careful
            to have the direction in the same ``level-wise'' slice.
      \item Notice that, with some special attention to what operation  we are
            carrying our, this is the exact same definition as classic
            multivariable calculus.
      \item There is another formulation that is (generally) more useful
            for computation
    \end{itemize}
  }
\end{frame}

\begin{frame}{Example: \(g(x)=x^3\)}
  \begin{align*}
    \onslide<+->{g(X+tH) &=}
    \onslide<+->{X^3+ tX^2H +tXHX + t^2XH^2\\
    & \hphantom{= X^3}+ tHX^2 + t^2HXH + t^2H^2X +t^3H^3.}
  \end{align*}
  \onslide<+->{
  From here, we can calculate:}
  \begin{align*}
    \onslide<+->{\frac{d}{dt} g(X+tH) &= X^2H + XHX + 2tXH^2 +HX^2\\
                  &\hphantom{=X^2H} +2tHXH +2tH^2X + 3t^2H^3 }\\
	\vspace{\stretch{2}}
    \onslide<+->{\frac{d^2}{dt^2} g(X+tH) &= 2XH^2+2HXH +2H^2X + 6tH^3 \\}
	\vspace{\stretch{2}}
    \onslide<+->{\frac{d^3}{dt^3} g(X+tH) &= 6H^3.}
	\vspace{\stretch{2}}
  \end{align*}
  \note{
    \begin{itemize}
      \item Now we consider an example. Since \(Df(X)[H]\) is linear, we can
            just work with a single monomial
      \item First we expand \((x+th)^3\)---but we can't use the binomial
            theorem since \(x\) and \(h\) don't commute
      \item Once we expand, we take standard derivatives w.r.t \(t\)---treating
            \(X\) and \(H\) as formal symbols.
    \end{itemize}
  }
\end{frame}

\begin{frame}{Example: \(g(x)=x^3\)}
  \onslide<+->{And so the first 3 directional derivatives are:}
  \begin{align*}
    \onslide<+->{Df(X)[H] &= X^2H + XHX +HX^2\\}
	  \vspace{\stretch{2}} \\
    \onslide<+->{D^{(2)}f(X)[H] &= 2XH^2+2HXH +2H^2X \\}
	  \vspace{\stretch{2}} \\
    \onslide<+->{D^{(3)}f(X)[H] &= 6H^3}
	  \vspace{\stretch{2}}
  \end{align*}
\end{frame}

\subsection{Multivarible Functions}

\begin{frame}{How should we treat multivariable functions?}
  \onslide<+->{What about \(f(x,y) = xy \in \CC [x,y]\)? For
    \(X,Y \in M_n(\CC)\), what is \(f(X,Y)\)?}
  \[
    \onslide<+->{XY} \qquad
    \onslide<+->{YX} \qquad
    \onslide<+->{\frac{1}{2}(XY+YX)} \qquad
  \]
  \onslide<+->{Clearly \(\CC [x,y]\) is the wrong space!}
\end{frame}

\begin{frame}{nc Polynomials}
  \onslide<+->{We must construct a newspace!}
  \begin{itemize}
    \item<+-> Let \(x=(x_1, \dots, x_d)\) be a tuple of formal variables.
    \item<+-> A \textbf{word} in \(x\) is a product of these variables.
    \begin{itemize}
       \item<+-> e.g.\ \(x_1x_3x_1x_4^2\qquad x_2^4x_5^3\)
    \end{itemize}
    \item<+-> An \textbf{nc polynomial} is a linear combination of words in
          \(x\) over your favorite field.
  \end{itemize}
  \onslide<+->{Let \(\RR \langle x \rangle \) and \(\CC \langle x \rangle \)
    denote the set of nc polynomials over \(\RR \) and \(\CC \).}
\end{frame}

\begin{frame}{Example: \(f(x,y) = x^2-xyx-1 \in \RR \langle x,y \rangle \)}
  \[
  \onslide<+->{  X = \begin{bmatrix} 4 &2\\2&2 \end{bmatrix}  \qquad \text{ and } \qquad Y =\begin{bmatrix} 2&0\\0&0 \end{bmatrix}}
  \]
  \begin{align*}
  \onslide<+->{  f(X,Y) &= X^2 - XYX + I_2 \\}
  \onslide<+->{         &= \begin{bmatrix} 4 &2\\2&2 \end{bmatrix}^2
             -\begin{bmatrix} 4 &2\\2&2 \end{bmatrix}\begin{bmatrix} 2&0\\0&0 \end{bmatrix}\begin{bmatrix} 4 &2\\2&2 \end{bmatrix}
             +\begin{bmatrix} 1&0\\0&1 \end{bmatrix} \\}
  \onslide<+->{         &= \begin{bmatrix} -11&-4\\-4&1 \end{bmatrix}.}
  \end{align*}
\end{frame}

\begin{frame}{Example: \(f(x,y) = x^2-xyx-1 \in \RR \langle x,y \rangle \)}
  \[
  \onslide<+->{  X = \begin{bmatrix} 4 &2\\2&2 \end{bmatrix}  \qquad \text{ and } \qquad Y =\begin{bmatrix} 2&0\\0&0 \end{bmatrix}}
  \]
  \begin{align*}
    \onslide<+->{  f(X\oplus X,Y\oplus Y) &=}
    \onslide<+->{\begin{bmatrix} -11&-4&0&0\\-4&1&0&0 \\ 0&0&-11&-4 \\ 0&0&-4&1 \end{bmatrix} \\}
    \onslide<+->{&= f(X,Y) \oplus f(X,Y).}
  \end{align*}
  \note{
    \begin{itemize}
      \item Mention that directional derivatives still work
      \item nc polynomials also do the unitary thing --- that is worth mentioning
    \end{itemize}
  }
\end{frame}

\subsection{Matrix Universe}

\begin{frame}
  \begin{center}
    \textcolor{goodgood}{\huge{Objects and Maps}} \\
    \textcolor{goodgood2}{\large{A Second Attempt}} \\
  \end{center}
  \note[item]{Second attempt takes polynomials as an example object instead of
    as a thing to lift}
\end{frame}

\begin{frame}{Some Definitions}
  \onslide<+->{
  \begin{Definition}[Matrix Universe]
    The \(g\)-dimensional \textbf{Matrix Universe} is
    \[
      \MM^{g} = \bigcup_{n \in \NN } (M_n(\CC))^g
    \]
  \end{Definition}}
  \onslide<+->{By convention, \(X \in \MM^{g}\) is a tuple of like-size matrices}
\end{frame}

\begin{frame}
  \onslide<+->{
  \begin{Definition}[Free Set]
  We say \(D \subset \MM^g\) is a \textbf{free set} if it is closed with respect
  to direct sums and unitary conjugation. That is,}
  \begin{enumerate}
    \item<+-> \(X,Y \in D \) means \(X\oplus Y \in D\).
    \item<+-> For \(X,U\) like-size matrices with \(U\) unitary and \(X \in D\),
          then \(U X U^* = \left( UX_1U^*, \dots , UX_g U^*  \right) \in D \).
  \end{enumerate}
  \end{Definition}
  \note{
    \begin{itemize}
      \item {\color{blue} should I write the defn for fiber and envelope or just
            say it verbally}
    \end{itemize}
  }
\end{frame}

\end{document}
%%% Local Variables:
%%% mode: latex
%%% TeX-engine: luatex
%%% End:
