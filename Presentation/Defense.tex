\documentclass[xcolor=dvipsnames, notes]{beamer}

%%%%%%%%%%%%%%%%%%%
% Packages/Macros %
%%%%%%%%%%%%%%%%%%%
\usepackage{amsmath,amsfonts,amssymb,amsthm, mathtools, mathrsfs, faktor, bbm}
\usepackage{wrapfig}
\usepackage{array}
\usepackage{graphicx}
\usepackage{tikz}
\usepackage{tikz-cd}
\usepackage{enumerate}

\usepackage{import}
\usepackage{xifthen}
\usepackage{pdfpages}
\usepackage{transparent}
\newcommand{\incfig}[1]{%
  \def\svgwidth{0.5\columnwidth}
  \import{./images/}{#1.pdf_tex}
}
\newcommand*\circled[1]{\tikz[baseline=(char.base)]{
            \node[shape=circle,draw,inner sep=2pt] (char) {#1};}}


\newcommand{\im}{\mathrm{Im}}
\newcommand{\re}{\mathrm{Re}}
\newcommand{\res}{\mathrm{Res}}
\newcommand{\pv}{\mathrm{p.v.}}
\DeclareMathOperator{\tr}{tr}
\DeclareMathOperator{\Log}{Log}
\DeclareMathOperator{\adj}{adj}
\DeclareMathOperator{\Hom}{Hom}
\DeclareMathOperator{\rk}{rk}
\DeclareMathOperator{\divv}{div}

% General shortcuts
\newcommand{\RR}{\mathbb{R}}
\newcommand{\CC}{\mathbb{C}}
\newcommand{\ZZ}{\mathbb{Z}}
\newcommand{\QQ}{\mathbb{Q}}
\newcommand{\NN}{\mathbb{N}}
\newcommand{\HH}{\mathbb{H}}
\newcommand{\MM}{\mathcal{M}}
\newcommand{\Id}{\mathbbm{1}}
\newcommand{\wms}{\textsc{wms} }
\newcommand{\wma}{\textsc{wma} }
\newcommand{\wwlog}{\textsc{wlog} }
\newcommand{\Arg}{\text{Arg}}
\renewcommand\qedsymbol{$\blacksquare$}


%\usepackage[utf8]{inputenc}
\usepackage{pgfpages}

\usepackage{amssymb,latexsym,amsmath,relsize,multicol, amsthm, tikz,tikz-cd, wrapfig, setspace,framed,xcolor,array , array, faktor, listings, mathrsfs , blkarray,booktabs,bigstrut}

\usepackage{soul}
\renewcommand<>{\hl}[1]{\only#2{\beameroriginal{\hl}}{#1}}

% https://tex.stackexchange.com/questions/41683/why-is-it-that-coloring-in-soul-in-beamer-is-not-visible
\makeatletter
\newcommand\SoulColor{%
  \let\set@color\beamerorig@set@color
  \let\reset@color\beamerorig@reset@color}
\makeatother
\SoulColor

\usetikzlibrary{arrows}

\usepackage[framemethod=TikZ]{mdframed}
\mdfsetup{nobreak=true}

\usepackage{stackengine}
\newcommand\xrowht[2][0]{\addstackgap[.5\dimexpr#2\relax]{\vphantom{#1}}}

\usepackage{hhline}

% \usepackage{tikzlings}

%%%%%%%%%%%%%%%%%%
% Beamer Options %
%%%%%%%%%%%%%%%%%%

%\usetheme{Frankfurt}
\usetheme{Antibes}
%\usetheme{AnnArbor}

% Color definitions:
\definecolor{mypurple}{RGB}{179,120,211}
\definecolor{darkpurple}{RGB}{130,98,168}
\definecolor{mygrey}{RGB}{164,167,172}
\definecolor{mygrey2}{RGB}{217,219,220}
\definecolor{fgreen}{RGB}{34,169,75}
\definecolor{gpurple}{RGB}{152,68,158}

%\useoutertheme{infolines} % Alternatively: miniframes, infolines, split
\useinnertheme{circles}
\definecolor{goodgood}{RGB}{127, 23, 52} % UBC Blue (primary)
\definecolor{goodgood1}{RGB}{46, 82, 102} % UBC Blue (primary)
\definecolor{goodgood2}{RGB}{203, 133, 137}
\usecolortheme[named=goodgood]{structure}
%\usecolortheme[named=Mahogany]{structure} % Sample dvipsnames color

%\setbeamercolor{structure}{fg=gpurple} % Title box and slide title box color
%\setbeamercolor{frametitle}{fg=white} % slide title text color
%\setbeamercolor{section in head/foot}{bg=gpurple} % frame background color
%\setbeamercolor{section in head/foot}{fg=white} % frame text color

\setbeamertemplate{navigation symbols}{} % disable navigation icons
\setbeamertemplate{items}[circle]

% Beamer theme font
\usefonttheme{serif}

% % --- page number ---
% \setbeamertemplate{footline}{%
% 	\raisebox{10pt}{\makebox[\paperwidth]{\hfill\makebox[7em]{\normalsize\texttt{\insertframenumber/\inserttotalframenumber}}}}%
% }

% Presenter's note
\setbeameroption{show notes on second screen}
%%%%%%%%%%%%%%%%%%%%%%%%%%%%%%
% Theorem/Proof Environments %
%%%%%%%%%%%%%%%%%%%%%%%%%%%%%%

\renewenvironment{definition}[1][]{\par\medskip\noindent \textbf{Definition:} \textit{#1} \rmfamily \\}{\vspace{4pt}}

\renewenvironment{example}[1][]{\par\medskip\noindent \textbf{Example.} \rmfamily}{\medskip}

\newenvironment{proposition}[1][]{\par\medskip\noindent \textbf{Proposition:} \rmfamily \\}{ \vspace{4pt}}

\newenvironment{mtheorem}[1][]{\begin{mdframed}[roundcorner=10pt] \refstepcounter{theorem}\par\medskip\noindent \textbf{Theorem~\thetheorem.} [#1] \rmfamily \\}{ \vspace{4pt}\end{mdframed}\vspace{-5pt}}
\numberwithin{theorem}{section}

\newenvironment{iproposition}[1][]{\begin{mdframed}[roundcorner=10pt] \refstepcounter{theorem}\par\medskip\noindent \textbf{Proposition~\thetheorem.} [#1] \rmfamily \\}{ \vspace{4pt}\end{mdframed}\vspace{-5pt}}
\numberwithin{theorem}{section}

\newenvironment{conjecture}[1][]{\begin{mdframed}[roundcorner=10pt] \refstepcounter{theorem}\par\medskip\noindent \textbf{Conjecture~\thetheorem.} \rmfamily \\}{ \vspace{4pt}\end{mdframed}\vspace{-5pt}}
\numberwithin{theorem}{section}

\newenvironment{nconjecture}[1][]{\begin{mdframed}[roundcorner=10pt] \refstepcounter{theorem}\par\medskip\noindent \textbf{Conjecture~\thetheorem.(#1)} \rmfamily \\}{ \vspace{4pt}\end{mdframed}\vspace{-5pt}}
\numberwithin{theorem}{section}

\theoremstyle{remark}
\newtheorem*{remark}{Remark}

%%%%%%%%%%%%%%%%%%%
% Custom Commands %
%%%%%%%%%%%%%%%%%%%

\newcommand{\paren}[1]{\left( #1 \right)}
\newcommand{\bracket}[1]{\left[ #1 \right]}
\newcommand{\cparen}[1]{\left\{ #1 \right\}}
\newcommand{\eval}[3]{\left. #1 \right|_{#2}^{#3}}
\newcommand{\abs}[1]{\left| #1 \right|}


%%%%%%%%%%%%
% Document %
%%%%%%%%%%%%

\title{Searching for Holes in the Matrix Universe}
\author{Lucas Kerbs}
\date{Spring 2022}

\begin{document}
\begin{frame}[plain]
  \maketitle
  \note{
    \begin{itemize}
      \item  Eventual goal: lift the tools of algebraic topology to spaces of
            matrices
      \item If we only consider \(2 \times 2\) matrices we can use classical
            theory
      \item The moment we want more than one size, things the classical theory
            breaks down
      \item Today we will develop some \emph{fairly heftly} tools to do just that
      \item Along the way, hopefully I can convince you that this is an
            interesting question.
      \item  To do so, we need to go back to our mathematical roots
    \end{itemize}
  }
\end{frame}

\section{Part I: Objects and Maps}

\begin{frame}
  \begin{center}
    \textcolor{goodgood}{\huge{Part I: Objects and Maps}} \\
    \textcolor{goodgood2}{\large{A Naive Attempt}} \\
  \end{center}
  \note{
    \begin{itemize}
      \item  That's right---objects and maps.
      \item Our Naive attempt involves that looking at lifting functions on
    \(\RR \) or \(\CC \) to accept matrices as their input.
      \item An operator theorist would call this a ``functional calculus''
    \end{itemize}
  }
\end{frame}

\subsection{Functional Calculus}

\begin{frame}{Functional Calculus}
  \onslide<1->{Let \(f \in \RR [x]\) and \(A \in M_k(\CC)\) be self adjoint.}\\
  \onslide<2->{\(A\) is diagonalizable as \(A = U \Lambda U^{*}\)}
  \begin{align*}
    \onslide<3->{f(A) &= a_nA^n + \cdots + a_1A + a_0 I_k  \\}
    \onslide<4->{&= a_n \left( U\Lambda U^* \right) ^n + \cdots + a_1 U\Lambda U^* + a_0 I_k  \\}
    \onslide<5->{&= a_n U\Lambda^n U^* + \cdots + a_1 U\Lambda U^* + a_0 I_k  \\}
    \onslide<6->{&= U \left( a_n\Lambda ^n + \cdots + a_1\Lambda + a_0 I_k \right) U^*  \\}
    \onslide<7->{&= U \left( f(\Lambda) \right) U^*}
  \end{align*}
  \[
   \onslide<8->{f \left( \begin{bmatrix} \lambda_1 &  &  \\  & \ddots &  \\  &  & \lambda_n \end{bmatrix}  \right)}
   \onslide<9->{ =\begin{bmatrix} f(\lambda_1) &  &  \\  & \ddots &  \\  &  & f(\lambda_n) \end{bmatrix}}
  \]
  \note{
    \begin{itemize}
    \item<+-> Polynomials are the most well behaved functions we have, so lets
            start with a polynomial and a self adjoint (\(A=A^*\)) matrix.
    \item<1-> You might say that SA is unnecessary bc we can already evaluate a
            polynomial on a matrix
    \item<+-> Since we are SA, we can diagonalize with unitary matrices.
    \item<+-> Watch what happens when we plug this into our polynomial --- need to be
            careful with the constant term.
    \item<9-> This application along the diagonal is precisely the behavior we
            want to emulate in the functional calculus.
    \end{itemize}
  }
\end{frame}

\begin{frame}
  \onslide<+->{
  Let \(\HH_n\) be the set of \(n\times n\) self adjoint matrices, and define
  \[
    \HH = \bigcup_{n \in \NN } \HH_n, \qquad \MM = \bigcup_{n \in \NN }M_n(\CC)
  \]}
  \onslide<+->{
  \begin{definition}
    Let \(g:[a,b]\to \CC \) and \(D \subset\HH \) denote the set of self adjoint
    matrices with their spectrum in \([a,b]\).}
    \onslide<+->{Then
    \begin{align*}
      g: D &\longrightarrow \MM  \\
         X=U\Lambda U^{*}  &\longmapsto U
                     \begin{bmatrix} g(\lambda_1) & &\\ &\ddots& \\ & & g(\lambda_n) \end{bmatrix}
                    U^*
    .\end{align*}}
  \end{definition}
  \onslide<+->{
  \noindent\textbf{Important:} In this functional calculus, \[g(X\oplus Y) = g(X) \oplus g(Y)\]}
  \note{
    \begin{itemize}
      \item With the polynomial case in mind, we can extend a general function.
         First, a piece of notation
      \item Lets grab a function on the real line and the self adjoint
            matrices with their spectrum in that domain
      \item Then we can lift \(g\) by emulating the behavior of polynomials.
      \item unwrap a self adjoint matrix, apply \(g\) to the diagonal, then wrap
            it back up
      \item Something to notice about this functional calculus---it treats
            direct sums \emph{very} well
      \item This is all well and good, but can we do anything with these new functions?
    \end{itemize}
  }
\end{frame}

\begin{frame}{Directional Derivative}
  \onslide<+->{
  \begin{Definition}[Directional Derivative]
    Fix some \(X \in \HH_n\). The derivative of \(f\) at \(X\) in the direction
    \(H \in M_n(\CC)\) is
    \[
      Df(X)[H] = \lim_{t \to 0} \frac{f(X+tH)-f(X)}{t}
    \]}
  \onslide<+->{
    Alternatively,
    \[
    Df(X)[H] = \left.\frac{df(X+tH)}{dt}\right|_{t=0}
    \]}
  \end{Definition}
  \note{
    \begin{itemize}
      \item We can define a directional derivative---as long as we are careful
            to have the direction in the same ``level-wise'' slice.
      \item Notice that, with some special attention to what operation  we are
            carrying our, this is the exact same definition as classic
            multivariable calculus.
      \item There is another formulation that is (generally) more useful
            for computation
    \end{itemize}
  }
\end{frame}

\begin{frame}{Example: \(g(x)=x^3\)}
  \begin{align*}
    \onslide<+->{g(X+tH) &=}
    \onslide<+->{X^3+ tX^2H +tXHX + t^2XH^2\\
    & \hphantom{= X^3}+ tHX^2 + t^2HXH + t^2H^2X +t^3H^3.}
  \end{align*}
  \onslide<+->{
  From here, we can calculate:}
  \begin{align*}
    \onslide<+->{\frac{d}{dt} g(X+tH) &= X^2H + XHX + 2tXH^2 +HX^2\\
                  &\hphantom{=X^2H} +2tHXH +2tH^2X + 3t^2H^3 }\\
	\vspace{\stretch{2}}
    \onslide<+->{\frac{d^2}{dt^2} g(X+tH) &= 2XH^2+2HXH +2H^2X + 6tH^3 \\}
	\vspace{\stretch{2}}
    \onslide<+->{\frac{d^3}{dt^3} g(X+tH) &= 6H^3.}
	\vspace{\stretch{2}}
  \end{align*}
  \note{
    \begin{itemize}
      \item Now we consider an example. Since \(Df(X)[H]\) is linear, we can
            just work with a single monomial
      \item First we expand \((x+th)^3\)---but we can't use the binomial
            theorem since \(x\) and \(h\) don't commute
      \item Once we expand, we take standard derivatives w.r.t \(t\)---treating
            \(X\) and \(H\) as formal symbols.
    \end{itemize}
  }
\end{frame}

\begin{frame}{Example: \(g(x)=x^3\)}
  \onslide<+->{And so the first 3 directional derivatives are:}
  \begin{align*}
    \onslide<+->{Df(X)[H] &= X^2H + XHX +HX^2\\}
	  \vspace{\stretch{2}} \\
    \onslide<+->{D^{(2)}f(X)[H] &= 2XH^2+2HXH +2H^2X \\}
	  \vspace{\stretch{2}} \\
    \onslide<+->{D^{(3)}f(X)[H] &= 6H^3}
	  \vspace{\stretch{2}}
  \end{align*}
  \note{
    \begin{itemize}
      \item Now all we have to do is set \(t=0\) in the previous expressions and
            we get the first three directional derivatives
      \item Note that they are all in the same direction---mixes derivatives are
            possible but we won't need them.
    \end{itemize}
  }
\end{frame}

% \subsection{Multivarible Functions} %

% \begin{frame}{How should we treat multivariable functions?}
%   \onslide<+->{What about \(f(x,y) = xy \in \CC [x,y]\)? For
%     \(X,Y \in M_n(\CC)\), what is \(f(X,Y)\)?}
%   \[
%     \onslide<+->{XY} \qquad
%     \onslide<+->{YX} \qquad
%     \onslide<+->{\frac{1}{2}(XY+YX)} \qquad
%   \]
%   \onslide<+->{Clearly \(\CC [x,y]\) is the wrong space!}
% \end{frame}

% \begin{frame}{nc Polynomials}
%   \onslide<+->{We must construct a newspace!}
%   \begin{itemize}
%     \item<+-> Let \(x=(x_1, \dots, x_d)\) be a tuple of formal variables.
%     \item<+-> A \textbf{word} in \(x\) is a product of these variables.
%     \begin{itemize}
%        \item<+-> e.g.\ \(x_1x_3x_1x_4^2\qquad x_2^4x_5^3\)
%     \end{itemize}
%     \item<+-> An \textbf{nc polynomial} is a linear combination of words in
%           \(x\) over your favorite field.
%   \end{itemize}
%   \onslide<+->{Let \(\RR \langle x \rangle \) and \(\CC \langle x \rangle \)
%     denote the set of nc polynomials over \(\RR \) and \(\CC \).}
% \end{frame}

% \begin{frame}{Example: \(f(x,y) = x^2-xyx-1 \in \RR \langle x,y \rangle \)}
%   \[
%   \onslide<+->{  X = \begin{bmatrix} 4 &2\\2&2 \end{bmatrix}  \qquad \text{ and } \qquad Y =\begin{bmatrix} 2&0\\0&0 \end{bmatrix}}
%   \]
%   \begin{align*}
%   \onslide<+->{  f(X,Y) &= X^2 - XYX + I_2 \\}
%   \onslide<+->{         &= \begin{bmatrix} 4 &2\\2&2 \end{bmatrix}^2
%              -\begin{bmatrix} 4 &2\\2&2 \end{bmatrix}\begin{bmatrix} 2&0\\0&0 \end{bmatrix}\begin{bmatrix} 4 &2\\2&2 \end{bmatrix}
%              +\begin{bmatrix} 1&0\\0&1 \end{bmatrix} \\}
%   \onslide<+->{         &= \begin{bmatrix} -11&-4\\-4&1 \end{bmatrix}.}
%   \end{align*}
% \end{frame}

% \begin{frame}{Example: \(f(x,y) = x^2-xyx-1 \in \RR \langle x,y \rangle \)}
%   \[
%   \onslide<+->{  X = \begin{bmatrix} 4 &2\\2&2 \end{bmatrix}  \qquad \text{ and } \qquad Y =\begin{bmatrix} 2&0\\0&0 \end{bmatrix}}
%   \]
%   \begin{align*}
%     \onslide<+->{  f(X\oplus X,Y\oplus Y) &=}
%     \onslide<+->{\begin{bmatrix} -11&-4&0&0\\-4&1&0&0 \\ 0&0&-11&-4 \\ 0&0&-4&1 \end{bmatrix} \\}
%     \onslide<+->{&= f(X,Y) \oplus f(X,Y).}
%   \end{align*}
%   \note{
%     \begin{itemize}
%       \item Mention that directional derivatives still work
%       \item nc polynomials also do the unitary thing --- that is worth mentioning
%     \end{itemize}
%   }
% \end{frame}

\subsection{Matrix Universe}

\begin{frame}
  \begin{center}
    \textcolor{goodgood}{\huge{Part I.5: Objects and Maps}} \\
    \textcolor{goodgood2}{\large{A Second Attempt}} \\
  \end{center}
  \note{
    \begin{itemize}
      \item In seeking a more general theory we need to leave the world of this
            ``SA functional calculus'' behind.
      \item Rather than lifting functions to be matrix valued, we will define
            \emph{new} objects that behave like those we just looked at.
    \end{itemize}
  }
\end{frame}

\begin{frame}{Some Definitions}
  \onslide<+->{
  \begin{Definition}[Matrix Universe]
    The \(g\)-dimensional \textbf{Matrix Universe} is
    \[
      \MM^{g} = \bigcup_{n \in \NN } (M_n(\CC))^g
    \]
  \end{Definition}}
  \onslide<+->{By convention, \(X \in \MM^{g}\) is a tuple of like-size matrices}

  \note{
    \begin{itemize}
      \item Definition of the matrix universe --- \(g\) tuples of matrices of
            all sizes
      \item By convetion, tuples are likes size
    \end{itemize}
  }
\end{frame}

\begin{frame}
  \onslide<+->{
  \begin{Definition}[Free Set]
  We say \(D \subset \MM^g\) is a \textbf{free set} if it is closed with respect
  to direct sums and unitary conjugation. That is,}
  \begin{enumerate}
    \item<+-> \(X,Y \in D \) means
          \(X\oplus Y = (X_1\oplus Y_1, \dots, X_g\oplus Y_g)\in D\).
    \item<+-> For \(X,U\) like-size matrices with \(U\) unitary and \(X \in D\),
          then \(U X U^* = \left( UX_1U^*, \dots , UX_g U^*  \right) \in D \).
  \end{enumerate}
  \onslide<+->{For \(D\) a free set, define \(D_n = D \cap (M_n(\CC))^g \)}.
  \end{Definition}
  \note{
    \begin{itemize}
      \item In math we often think about substructures that
            capture the implicit structure our space (subgroup, subspace, etc)
      \item In the nc setting, this is a \emph{free set}, also called nc set
      \item direct sums and unitary conjugation are component wise
      \item If you see a \(D\), you can assume that it is a free set.
      \item A subscript denotes a level-wise slice
      \item Note that this requires a lot of structure on free sets---we want to
            put a name to these structure.
    \end{itemize}
  }
\end{frame}

\begin{frame}
  \onslide<+->{
  \begin{Definition}[Fiber]
    Given \(X \in \MM^{g} \), a tuple of \(n \times n\) matrices, the
    \textbf{fiber} of \(X\) is the set
    \[
      \{X^{\oplus k} \mid  k \in \NN \}.
    \]
  \end{Definition}}
  \onslide<+->{
  \begin{Definition}[Envelope]
    Given \(X \in \MM^{g} \), a tuple of \(n \times n\) matrices, the
    \textbf{envelope} of \(X\) is the set
    \[
      \{ U^* X^{\oplus k} U \mid k \in \NN, U \text{ Unitary} \}.
    \]
  \end{Definition}}
  \note{
    \begin{itemize}
      \item The fiber is all the points ``above'' a given point. Conceptually,
            we imagine identification along the fiber---this will become
            important when we start doing topology
      \item The envelope (which will be less important to us) is the unitary
            smearing of the fiber at each level.
    \end{itemize}
  }
\end{frame}

\begin{frame}{A bit of topology}
  \onslide<+->{}
  \onslide<+->{What does it mean for \(D \subset \MM^{g} \) to be open?}
  \begin{itemize}
    \item<+-> There is not a canonical topology
    \item<+-> For us, we will say that \(D\) is open if each \(D_n\) is open.
    \item<+-> Simply connected, connected, bounded, etc.\ are defined similarly.
  \end{itemize}
  \note{
    \begin{itemize}
      \item If we are going to look for holes and build up the algebraic
            topology, we need a point set topology first---so there is a natural
            question.
      \item Bad news: there isn't a natural choice
      \item There are a handful of candidates (fine, fat, free, nc Zariski). I
            wish we had time to go into detail.
      \item For us, free sets are open if their level-wise restriction is open
      \item Other point-set characterizations are similar.
    \end{itemize}
  }
\end{frame}

\begin{frame}{What the natural functions on \(\MM^{g} \)?}
  \onslide<+->{}
  \onslide<+->{
  \begin{Definition}
  A function \(f: D\to \MM^{\hat{g}}\) is called \textbf{free} if
  \begin{enumerate}
    \item \(f(X\oplus Y)= f(X) \oplus f(Y)\)
    \item \(f(U X U^*) = f(U)f(X)f(U^*)\) where \(X\) and \(U\) are like-size
          and \(U\) is unitary.
  \end{enumerate}
  \end{Definition}}
  \onslide<+->{
  \begin{Definition}
  A function \(f: D \to \CC \) is a \textbf{tracial function} if
  \begin{enumerate}
    \item \(f(X\oplus Y) = f(X)+f(Y)\)
    \item \(f(U X U^*) = f(X)\) where \(X\) and \(U\) are like-size
          and \(U\) is unitary.
  \end{enumerate}
  \end{Definition}}
  \note{
    \begin{itemize}
      \item We have our objects, but what are the maps?
      \item Free functions are defined to be anything that behaves like a
            polynomial
      \item tracial functions look like traces
      \item For both of these maps, the directional derivative is defined
            identically as before---but tracial functions get something extra.
    \end{itemize}
  }
\end{frame}

\subsection{Uniqueness of the Gradient}

\begin{frame}
  \begin{Definition}[Free Gradient]
    Given a tracial function \(f\), the free gradient, \(\nabla f\), is the
    unique free function satisfying
    \[
      \tr \left( H \cdot \nabla f(X) \right) = Df(X)[H],
    \]
    where, if \(A= (A_1, \dots, A_g)\) and \(B=(B_1, \dots, B_g)\) are tuples of
    like-size matrices then \(A \cdot B = \sum_{i=1}^g A_i B_i \).
  \end{Definition}
  \note{
    \begin{itemize}
      \item The \(\nabla\) of a free function is the unique free function
            satisfy this equation---where the * is just like the dot product
      \item Whenever you see tr(*) I want you to think of the inner product---it
            is slightly distinct but it will make a lot of things make more
            sense
      \item Some of you may be hesitant at the fact that I claim \(\nabla\) is
            unique. Why should this be true?
    \end{itemize}
  }
\end{frame}

\begin{frame}{Why should \(\nabla f\) be unique?}
  \onslide<+->{
  \begin{theorem}[Trace Duality]
  Let \(f,g\) be free functions \(\MM^{g} \to \MM^{\tilde{g}} \). If
  \(\tr (H \cdot f) = \tr (H \cdot g)\) for all tuples \(H\), then \(f=g\).
  \end{theorem}}
  \note{
    \begin{itemize}
      \item \(f=g\) whenever the domains overlap
      \item In the vector space setting---with an inner product---this is a
            fairly immediate result. You would show it by picking vectors of all
            0's and a single 1.
      \item You prove this identically---but with coordinate matrices instead of
            coordinate vectors.
    \end{itemize}
  }
\end{frame}

%\begin{frame}
  %\emph{Proof.}
  %%\onslide<+->{Let \(E_{ij}\) be a matrix of all \(0\)'s with a \(1\) in the
    %\(i,j\) slot. }
%
  %\onslide<+->{Let \(H = (E_{ji}, 0, \dots, 0)\). }
%
  %There isn't time for this proof.
%\end{frame}

\section{Part II: Monodromy}

\begin{frame}
  \begin{center}
    \textcolor{goodgood}{\huge{Part II: Analytic Continuation and Monodromy}} \\
  \end{center}
\end{frame}

\subsection{Analytic Continuation}

\begin{frame}{Analytic Continuation}
  Figures of generic analytic continuation
\end{frame}

\begin{frame}
  figures of the example of continuing log(x) through UHP and LHP
\end{frame}

\subsection{Monodromy}

\begin{frame}{When are two analytic continuations equal?}
  \onslide<+->{
  \begin{theorem}[Monodromy I]%
    Let \(\gamma_1, \gamma_2\) be two paths from \(\alpha\) to \(\beta\) and
    \(\Gamma_s\) be a fixed-endpoint homotopy between them. If \(f\) can be
    continued along \(\Gamma_s\) for all \(s \in [0,1]\), then the continuations
    along \(\gamma_1\) and \(\gamma_2\) agree at \(\beta\).
  \end{theorem}}
\onslide<+->{
  figure of the homotopy}
\end{frame}

\begin{frame}{When are two analytic continuations equal?}
  \onslide<+->{
  \begin{theorem}[Monodromy II]%
    Let \(U \subset \CC \) be a disk in \(\CC \) centered at \(z_0\) and
    \(f: U \to \CC \) an analytic function. If \(W\) is
    an open, simply connected set containing \(U\) and \(f\) continues along any
    path \(\gamma \subset W\) starting at \(z_0\), then \(f\) has a unique
    extension to all of \(W\).
  \end{theorem}}

  \onslide<+->{figure of the simply connected larger set}
\end{frame}

\subsection{Free Monodromy}

\begin{frame}{What about the nc case?}
  \onslide<+->{
  Before we look at a free analogue of the monodromy theorem, we need to ask an
  important question: What does it mean for a free function to be analytic?}

  \vspace{\stretch{1}}
  \onslide<+->{\textbf{A}: \(f:D\to \MM^{\hat{g}}\) is analytic if it is
    analytic as a function of each \(D_n\).}

  \vspace{\stretch{1}}
  \onslide<+->{
  \begin{theorem}[Agler, McCarthy (2016)]
    Let \(f:D\to \MM^{\hat{g}}\) be a free function. If \(f\) is locally bounded
    on each \(D_n\), then \(f\) is an analytic free function.
  \end{theorem}}
  \note{
    \begin{itemize}
      \item be sure to comment on the theorem---it is powerful!
    \end{itemize}
  }
\end{frame}

\begin{frame}
  \onslide<+->{\textbf{Q:} What about the nc case? Can we say anything similar?}
  \vspace{\stretch{1}}
  \onslide<+->{
  \begin{theorem}[Free Universal Monodromy, Pascoe 2020]%
    Let \(f\) be an analytic free function defined on some ball \(B \subset D\),
    for \(D\) an open, connected free set.
    Then \(f\) analytically continues along every path in \(D\) if and only if
    \(f\) has a unique analytic continuation to all of \(D\).
  \end{theorem}}
  \vspace{\stretch{1}}
\end{frame}

\begin{frame}{Consequences of Free Monodromy}
  \onslide<+->{}
          \vspace{\stretch{2}}
  \begin{enumerate}
    \item<+-> Free functions can't detect holes!
          \vspace{\stretch{1}}
    \item<+-> If we want a fundamental group governed by analytic continuation,
      we need to look elsewhere.
          \vspace{\stretch{2}}
  \end{enumerate}
\end{frame}

\section{Part III: Homotopy}

\begin{frame}
  \begin{center}
    \textcolor{goodgood}{\huge{Part III: Homotopy}} \\
  \end{center}
\end{frame}

\subsection{A First Fundamental Group}

\begin{frame}
  \onslide<+->{
  \begin{definition}
    A continuous function \(\gamma:[0,1]\to D\) \textbf{essentially takes} \(X\) to \(Y\) if
    \begin{align*}
      \gamma(0) = X^{\oplus \ell},& \;\text{ for some \(\ell \in \NN \)}\\
      \gamma(1) = Y^{\oplus k},& \;\text{ for some \(k \in \NN \)}.
    \end{align*}
  \end{definition}}
  \onslide<+->{
  \begin{figure}[h!]
  \centering
    \def\svgwidth{0.7\columnwidth}
    \import{./img/}{essentialpath_talk.pdf_tex}
  \end{figure}}
\end{frame}

\begin{frame}
  \onslide<+->{
  Given \(\gamma\) essentially taking \(X\) to \(Y\) and \(\beta\) taking \(Z\)
  to \(W\), define
  \[
    \gamma\oplus\beta(t) := \begin{bmatrix} \gamma(t)&0\\0&\beta(t) \end{bmatrix}.
  \]}
  \onslide<+->{
  \begin{definition}%
    Let \(\gamma\) and \(\beta\) be paths taking \(X\) to \(Y\) and \(Y\) to \(Z\)
    respectively. We define their product to be the path essentially taking \(X\)
    to \(Z\) given by
    \[
      \beta\gamma(t) :=
      \begin{cases}
        \gamma^{\oplus k}(2t) & t \in [0,0.5) \\
        \beta^{\oplus\ell} (2t-1)& t \in [0.5,1]
      \end{cases}
    \]
    where \(k\) and \(\ell\) are positive integers chosen to make
    \(\gamma^{\oplus k}\) and \(\beta^{\oplus \ell}\) like size matrices for each
      \(t \in [0,1]\).
  \end{definition}}
\end{frame}

\begin{frame}{The Full Fundamental Group}
  For \(D \subset \MM^{g}\) a connected free set, the
  \textbf{full fundamenal group}, \(\pi_1(D)\), is the group of paths essentially taking \(X\)
  to \(X\) up to homotopy equivalence and the relation
  \(\gamma = \gamma^{\oplus k}\).
  \note{
    \begin{itemize}
      \item remark about how this misses the link to analytic continuation of
            functions, but free functions won't do.
    \end{itemize}
  }
\end{frame}

\subsection{A Second Fundamental Group}

\begin{frame}
  \onslide<+->{
  Let \(D \subset \MM^{g} \) be a connected, open, free set. If there exists a
  nonempty, simply-connected, open, free \(B \subset D\), then we say that \(D\)
  is \textbf{anchored}.}

  \vspace{\stretch{1}}
  \onslide<+->{
  For \(D\) an anchored set, and \(B \subset D\) its anchor, we call a tracial
  function \(f:B\to \CC \) a \textbf{global germ} if it analytically continues
  along every path in \(D\) which starts in \(B\).}

  \vspace{\stretch{1}}
  \onslide<+->{
  For our purposes, we view \(\gamma\) as coupled with its endpoint. Thus, if
  \(\gamma\) essentially takes \(X\) to \(Y\), then
  \[
    f(\gamma) = \frac{1}{k} f(Y^{\oplus k}).
  \]}
\end{frame}

\begin{frame}{Trace Equivalence}
  \onslide<+->{
  \begin{definition}
    Let \(B \subset D\) be an anchor and fix \(X \in B\). If \(\gamma\) and
    \(\beta\) both essentially take \(X\) to \(Y\), we say they are
    \textbf{trace equivalent} if, for every global germ \(f\) and every path
    \(\delta\) taking \(Y\) to \(Z\), \[f(\delta\gamma)=f(\delta\beta).\]}

    \onslide<+->{
    That is, trace equivalent paths are those which cannot be told apart via
    analytic continuation of global germ.
  \end{definition}}
\end{frame}

\begin{frame}{The Tracial Fundamental Group}
  \onslide<+->{}
  \onslide<+->{
  Let \(D \subset \MM^{g}\) be an anchored space with \(B\) is anchor.
  For \(X \in B\) define
  \(\pi_1^{\textrm{tr}}(D)\) to be the group of trace equivalent paths
  essentially taking \(X\) to \(X\).}
  \vspace{\stretch{1}}
  \onslide<+->{Computationally, we are still stuck.}
  \note{
    \begin{itemize}
      \item  remark that the choice of base point is irrelevant. and that it is
            a quotient of the full fundamental group
    \end{itemize}
  }
\end{frame}

\section{Part IV: Cohomology}%

\begin{frame}
  \begin{center}
    \textcolor{goodgood}{\huge{Part IV: Cohomology}} \\
  \end{center}
\end{frame}

\subsection{A Short Review of Cohomology}

\begin{frame}
  \onslide<+->{}
  \onslide<+->{Traditional homology considers a complex of the form
  \[
    \cdots
    \xleftarrow{\partial_{n-1}}
    C_{n-1}
    \xleftarrow{\partial_{n}}
    C_{n}
    \xleftarrow{\partial_{n+1}}
    C_{n+1}
    \xleftarrow{\partial_{n+2}}
    \cdots
  \]}
  \onslide<+->{
  While cohomology considers a dual complex
  \[
    \cdots
    \xrightarrow{d_{n-2}}
    C^{n-1}
    \xrightarrow{d_{n-1}}
    C^{n}
    \xrightarrow{d_{n}}
    C^{n+1}
    \xrightarrow{d_{n+1}}
    \cdots
  \]}

  \vspace{\stretch{1}}

  \onslide<+->{In general, \(C^k\) is a group of functions into some abelian
    group.}

  \vspace{\stretch{1}}

  \onslide<+->{The \(k\)-th cohomology group is
  \[
    H^k= \frac{\textrm{ker}\, d_i}{\textrm{Im}\, d_{i-1}}
  \]}
  \vspace{\stretch{1}}
\end{frame}

\subsection{Tracial Cohomology}

\begin{frame}
  \onslide<+->{
  Let \(D\) be an anchored set. Denote the set of
  (globally defined)
  tracial functions on \(D\) by \(\mathcal{T}(D)\) and the set of free functions by
  \(\mathcal{F}(D)\).}
  \onslide<+->{
  For \( f \in \mathcal{T}(D)\), \(\nabla f \in \mathcal{F}(D)\)---so we have the
  beginnings of a chain complex!
  \[
    0 \rightarrow \mathcal{T}(D) \xrightarrow{\nabla} \mathcal{F}(D) \rightarrow \cdots.
  \]}
\end{frame}

\begin{frame}
  \onslide<+->{
  \[
    0 \rightarrow \mathcal{T}(D) \xrightarrow{\nabla} \mathcal{F}(D) \rightarrow \cdots.
  \]}

  \onslide<+->{
  A free function \(g: D \to \MM^{g} \) is \textbf{exact} if there exists a
  tracial function \(f: D \to \CC \)
  such that \(\nabla f = g\).}

  \vspace{\stretch{1}}
  \onslide<+->{
  A free function \(g: D \to \MM^{g} \) is \textbf{closed} if
  \[
    \tr \left( K \cdot Dg(X)[H] \right) = \tr \left( H \cdot Dg(X)[K] \right)
  \]
  for all directions \(H,K\).
  }

  \vspace{\stretch{1}}
  \onslide<+->{
  \begin{definition}
    The \textbf{first tracial cohomology group} is the vector space of closed free
    functions moduluo the exact free function. We write \(H^1_{\textrm{tr}}(D) \).
  \end{definition}
  }
  \note{
    \begin{itemize}
      \item test text
    \end{itemize}
  }
\end{frame}

\begin{frame}{What out global germs?}
  \begin{itemize}
    \item<+-> For \(f: B \to \CC \) a global germ, since \(f\) analytically
          continues along every path, so does \(\nabla f\).
    \item<+-> Free Monodromy means that \(\nabla f\) has a global extension.
    \item<+-> \textbf{Important:} If \(f\) is a global germ, then \(\nabla f\)
          is not necessarily exact since \(\mathcal{T}(D)\) is the set of
          tracial functions defined on \emph{all of} \(D\).
  \end{itemize}
\end{frame}

\subsection{Injecting into \(\CC \)}

\begin{frame}
  \vspace{\stretch{1}}
  \onslide<+->{
  \textbf{Goal:} Show that \(\pi_1^{\tr}(D)\) injects into \(\CC \).}

  \vspace{\stretch{1}}
  \onslide<+->{
  \begin{lemma}%
    Let \(D\) be an anchored nc domain. For any \(\alpha,\beta \in \pi_1^{\tr}(D)\) and
    global germ \(f\),
    \[
      f(\alpha\beta) - f(\alpha) = f(\beta) - f(\tau)
    \]
    where \(\tau\) is the constant path.
  \end{lemma}}
  \vspace{\stretch{1}}
\end{frame}

\begin{frame}
  \onslide<+->{
  For \(D\) an anchored set, \(X \in B_1\) the base point, \(f\) a global germ,
  and \(\gamma \in \pi_1^{\tr}(D)\), define
  \[
    c^f(\gamma) = f(\gamma)-f(\tau)
  \]}

  \onslide<+->{\(c^f\) gives us a homomorphism into \(\CC \)!}
  \begin{align*}
    \onslide<+->{c^f(\gamma_1\gamma_2) &= f(\gamma_1\gamma_2) - f(\tau) \\}
               \onslide<+->{&= f(\gamma_1\gamma_2) - f(\gamma_1) + f(\gamma_1) -f(\tau) \\}
               \onslide<+->{&= f(\gamma_2) - f(\tau) + f(\gamma_1) -f(\tau) \\}
               \onslide<+->{&= c^f(\gamma_2) + c^f(\gamma_1).}
  \end{align*}
\end{frame}

\begin{frame}
  \vspace{\stretch{1}}

  \onslide<+->{
  \begin{lemma}
    The map
    \begin{align*}
	    \Phi: \prod_{\substack{\nabla f \in H^1_{\tr}(D) \\ f \textrm{ a global germ}}} \pi_1^{\tr}(D)
      &\longrightarrow \prod_{\substack{\nabla f \in H^1_{\tr}(D) \\ f \textrm{ a global germ}}}\CC  \\
      \prod \gamma &\longmapsto \prod c^f(\gamma)
    \end{align*}
    is an injective homomophism.
  \end{lemma}}

  \vspace{\stretch{1}}
  \onslide<+->{So \(\pi_1^{\tr}(D)\) is commutative and torsion free!}

  \vspace{\stretch{1}}
\end{frame}

\subsection{Characterizing \(\pi_1^{\tr}(D)\)}

\begin{frame}{\(\pi_1^{\tr}(D)\) is divisible}
  \onslide<+->{}
  \onslide<+->{
  For any \(\gamma \in \pi_1^{\tr}(D)\),
  \[
    \gamma\oplus\tau = \tau\oplus\gamma.
  \]}

  \vspace{\stretch{1}}
  \onslide<+->{
  Why?
  \[
    H(t,\theta)  = \begin{bmatrix} \cos \theta & \sin \theta \\ -\sin \theta& \cos \theta \end{bmatrix}
    \left( \gamma \oplus \gamma_X \right)
   \begin{bmatrix} \cos \theta & \sin \theta \\ -\sin \theta& \cos \theta \end{bmatrix}^*
  \]
  is a homotopy between the paths.}
  \vspace{\stretch{1}}
\end{frame}

\begin{frame}
\begin{align*}
  \small
\onslide<+->{  \gamma &= \underbrace{\begin{bsmallmatrix} \gamma &&&\\ &\gamma&\\&&\ddots\\&&&\gamma\end{bsmallmatrix}}_{k+1\textrm{-times}}\\}
    \onslide<+->{&=\begin{bsmallmatrix} \gamma &&&\\ &\tau&\\&&\ddots\\&&&\tau\end{bsmallmatrix}
      \begin{bsmallmatrix} \tau &&&\\ &\gamma&\\&&\ddots\\&&&\tau\end{bsmallmatrix}\cdots
      \begin{bsmallmatrix} \tau &&&\\ &\tau&\\&&\ddots\\&&&\gamma\end{bsmallmatrix}\\}
    \onslide<+->{&=\begin{bsmallmatrix} \gamma &&&\\ &\tau&\\&&\ddots\\&&&\gamma_X\end{bsmallmatrix}
      \begin{bsmallmatrix} \gamma &&&\\ &\tau&\\&&\ddots\\&&&\tau\end{bsmallmatrix}\cdots
      \begin{bsmallmatrix} \gamma &&&\\ &\tau&\\&&\ddots\\&&&\tau\end{bsmallmatrix}\\}
    \onslide<+->{&=\begin{bsmallmatrix} \gamma &&&\\ &\tau&\\&&\ddots\\&&&\tau\end{bsmallmatrix}^k}
\end{align*}
\end{frame}

\begin{frame}
  \begin{theorem}
    For \(D\) an anchored free set, \(\pi_1^{\tr}(D)\) is a torsion free,
    abelian, divisible group. That is,
    \[
      \pi_1^{\tr}(D) \simeq \bigoplus_{i \in I} \QQ = \QQ ^I
    \]
    for some set \(I\).
  \end{theorem}
\end{frame}

\section{Computing \(\pi_1^{\tr}(D)\)}

\begin{frame}
  \begin{center}
    \textcolor{goodgood}{\huge{Part V: Computing \(\pi_1^{\tr}(D)\)}} \\
  \end{center}
  \note{
    \begin{itemize}
      \item hard bc no VC or MV
    \end{itemize}
  }
\end{frame}

\subsection{The Direct Limit Approach}

\begin{frame}
  \onslide<+->{
  Let \(D\) be an anchored, free, path connected set such that each \(D_n\) is
  nonempty and choose an anchor \(B \subset D\) such that each \(B_n\) is also
  nonempty. \\}


  \vspace{\stretch{1}}
  \onslide<+->{Let \(\pi_1^{\tr}(D)_n\) is the subgroup of paths in \(D_n\).
    Note that \(\pi_1^{\tr}(D)_1\) is a quotient of \(\pi_1(D_n)\).\\}


  \vspace{\stretch{1}}
  \onslide<+->{There is a natural inclusion
  \begin{align*}
      \pi_1^{\tr}(D)_n&\lhook\joinrel\longrightarrow  \pi_1^{\tr}(D)_{kn}\\
                  \gamma&\longmapsto \gamma^{\oplus k}
  \end{align*}
  for all \(k\).}
  \vspace{\stretch{1}}
\end{frame}

\begin{frame}
  \vspace{\stretch{2}}
  \onslide<+->{
  Now consider the chain of inclusions:
  \[
    \pi_1^{\tr}(D)_1 \hookrightarrow
    \pi_1^{\tr}(D)_2 \hookrightarrow
    \pi_1^{\tr}(D)_6 \hookrightarrow \cdots \hookrightarrow
    \pi_1^{\tr}(D)_{n!} \hookrightarrow \cdots
  \]}

  \vspace{\stretch{1}}
  \onslide<+->{The limit of this sequence isomorphic to \(\pi_1^{\tr}(D)\)!}

  \vspace{\stretch{2}}
\end{frame}

\subsection{An Example}

\begin{frame}{Example: \(\pi_1^{\tr}(GL)\)}
  \vspace{\stretch{1}}
  \onslide<+->{Let \(GL = \bigcup_{n \in \NN }GL_n(\CC)\).}

  \vspace{\stretch{1}}
  \onslide<+->{ \(GL_1(\CC) = \CC \setminus \{0\} \). Since
    \(\pi_1(GL_1) \simeq \ZZ \),
    \(\pi_1^{\tr}(GL)_1 \simeq \ZZ \)} as well.

  \vspace{\stretch{1}}
  \onslide<+->{Inclusion into \(\pi_1^{\tr}(GL)_2\) picks up square roots. If
  \(\gamma \in \pi_1^{\tr}(GL)_1\), then
  \[
    \begin{bmatrix} \gamma&\\&\tau \end{bmatrix} \in \pi_1^{\tr}(GL)_2.
  \]
  }

  \onslide<+->{Thus, \(\pi_1^{\tr}(GL)_2 \simeq \ZZ \left[\frac{1}{2}\right].\)}
  \vspace{\stretch{1}}
\end{frame}

\begin{frame}
  \vspace{\stretch{1}}
  \onslide<+->{Similarly, inclusion into \(\pi_1^{\tr}(GL)_{3!}\) picks up cube
    roots:
  \[
    \pi_1^{\tr}(GL)_6 \simeq \ZZ \left[\frac{1}{2}, \frac{1}{3}\right]
  \]
  }

  \vspace{\stretch{1}}
  \onslide<+->{In the \(n\)-th inclusion, we pick up \(n\)-th roots and so we adjoin \(\frac{1}{n}\) to the
    preceding group. Therefore,
  \[
    \pi_1^{\tr}(GL) \simeq \ZZ \left[\frac{1}{2},\frac{1}{3},\frac{1}{4}, \dots\right]}
    \onslide<+->{\simeq \QQ.}
  \]

  \vspace{\stretch{1}}
\end{frame}

\begin{frame}
  \begin{center}
    \textcolor{goodgood}{\huge{Thank You!}} \\
  \end{center}
\end{frame}



\end{document}
%%% Local Variables:
%%% mode: latex
%%% TeX-engine: luatex
%%% End:
