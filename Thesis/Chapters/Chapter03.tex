\chapter{Monodromy: Classical and Free}\label{ch:monodromy}
\section{Classical Monodromy}%
\label{sec:classmono}

In the study of functions of a single complex variable, many of the central
theorems surround the idea of analytic continuation. Given some analytic
function \(f\) on a domain \(\Omega \subset\CC \) and a larger domain
\(\overline{\Omega} \supset \Omega\), we can (with sufficient
``niceness'' conditions) extend \(f\) to an analytic function \(g\) on
\(\overline{\Omega}\). In particular, given some path \(\gamma\) which start in
\(\Omega\) we wish to continue \(f\) \emph{along} \(\gamma\) by recomputing the
power series on overlapping disks with their centers on \(\gamma\).
\begin{figure}[h!]
\centering
  \def\svgwidth{0.8\columnwidth}
  \import{./Chapters/img/part 2/}{monodromycurve.pdf_tex}
\caption{Analytic continuation along a curve}
\label{fig:monocurve}
\end{figure}

Our path \(\gamma\) must avoid any potential poles of \(f\) so that we may
compute the power series, but the uniqueness of such an extension is not
obvious. This is where the aforementioned niceness conditions come into play!
For example, consider the follow setup:

\begin{figure}[h!]
\centering
  \def\svgwidth{0.52\columnwidth}
  \import{./Chapters/img/part 2/}{monodromylog.pdf_tex}
\caption{Two paths in \(\CC \)}
\label{fig:monolog}
\end{figure}

If we let \(f(x)=\Log x\) be the principle branch of the complex logarithm the
defined on the right half plane, and continue \(f\) along \(\gamma_1\) and
\(\gamma_2\) we get two functions \(f_1\) and \(f_2\) which are analytic at
\(\beta\), but they don't agree! In the case, \(f_1(\beta)\) and \(f_2(\beta)\)
disagree by exactly \(2\pi i\). The monodromy theorem gives sufficient
conditions for the continuation along two curves to agree:

\begin{theorem}[Monodromy I]%
\label{thm:mon1}
  Let \(\gamma_1, \gamma_2\) be two paths from \(\alpha\) to \(\beta\) and
  \(\Gamma_s\) be a fixed-endpoint homotopy between them. If \(f\) can be
  continued along \(\Gamma_s\) for all \(s \in [0,1]\), then the continuations
  along \(\gamma_1\) and \(\gamma_2\) agree at \(\beta\).
\end{theorem}

In the example above, any homotopy between the two paths must pass through the
origin---where \(\Log x\) fails to be analytic---and hence the two continuations
disagree at \(\beta\).
An equivalent formulation of the monodromy theorem concerns extending a
functions to a larger domain:

\begin{theorem}[Monodromy II]%
\label{thm:mon2}
  Let \(U \subset \CC \) be a disk in \(\CC \) centered at \(z_0\) and
  \(f: U \to \CC \) an analytic function. If \(W\) is
  an open, simply connected set containing \(U\) and \(f\) continues along any
  path \(\gamma \subset W\) starting at \(z_0\), then \(f\) has a unique
  extension to all of \(W\).
\end{theorem}

This second formulation gives another perspective on \(\Log x\). In the example,
\(U\) is a disk around \(\alpha\) that stays in the right half plane and \(W\)
is \(\CC \setminus \{0\} \). While \(\Log x\) continues along any path in
\(\CC  \setminus \{0\} \), the larger domain is \emph{not} simply connected, so
monodromy fails.

In practice, after the initial exposure in a first course in complex variables,
no one computes continuations by hand. {\color{red} This could be a paragraph,
  but is it necessary?}

\section{Free Monodromy}%
\label{sec:freemono}

There is an analogous theorem to \cref{thm:mon1,thm:mon2} in the free settings
initial proven by J.E.\ Pasocoe in \cite{pascoeNoncommutative2020}.
In the classic case, the larger set \(W\) must be simply connected. In the free
setting, however, the theorem is much more powerful.

\begin{theorem}[Free Univeresal Monodromy]%
  \label{thm:freemono}
  If \(f\) is an analytic free function defined on some ball \(B \subset D\),
  for \(D\) an open, connected free set.
  Then \(f\) analytically continues along every path in \(D\) if and only if
  \(f\) has a unique analytic continuation to all of \(D\).
\end{theorem}

\begin{proof}
  [Proof (From \cite{pascoeNoncommutative2020})]
  The fact that a unique extension to all of \(D\) implies that \(f\) has a
  continuation along any \(\gamma\) is immediate.

  {\color{red} honestly I don't quite get the proof yet, so this will come later.}
\end{proof}
