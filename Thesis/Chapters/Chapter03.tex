%************************************************
\chapter{Zero Sets and Principle Divisors}\label{ch:ZeroDiv}
%************************************************

\section{Varieties, Classical and Free}%
\label{sec:varieties}

In the classical case, varieties are fairly easily to classify. Given some
(commutative) polynomial, \(f \in \CC [x_1, \dots, x_g]\) we define the zero set
\[
  V(f) = \{a \in \mathbb{A}^n \mid  f(a) =0\},
\]
where \(\mathbb{A}^n\) is complex affine \(n\)-space.  Varieties (both affine and
projective) are well studied in algebraic geometry
(Hartshorne's \emph{Alegbraic Geometry} \cite{hartshorneAlgebraic2008} is a
standard introduction). Of particular interest is a geometric invariant of a
variety called a \emph{divisor}. While divisors require robust machinery to
construct formally\footnote{Schemes, in particular.} one can think of them
(loosely) as formal sums of codimension one subvarieties. The concept of a
divisor lift naturally to the noncommutative setting, although varieties are
a touch more complex.

Let \(f\) be a matrix of polynomials on \(\MM^{g} \).
Unlike the classical case, it is not immediate what should be meant by
\(f(X)=0\)---is it enough for \(f(X)\) to be singular, or should \(f(X)\) be the
zero matrix? In light of this ambiguity, we make three definitions.
\begin{definition}[Singular Set]%
\label{def:singularset}
  Let \(f\) be a matrix of polynomials function. The \textbf{n-Singular Set} of \(f\) is
  \[
    \mathscr{Z}_n(f) = \{X \in M_n (\CC) \mid \det f(X) =0\}.
  \]
  The \textbf{Singular Set} of \(f\) is
  \[
    \mathscr{Z}(f) = \bigcup_{n \in \NN } \mathscr{Z}_n(f).
  \]
  Associated with the singular set is the \textbf{Directional Singular Set}:
  \[
    \mathscr{Z}_{\textrm{dir}}(f) = \{(X,v) \mid f(X)v = 0\}.
  \]
\end{definition}

\begin{definition}[Zero Set]%
\label{def:zeroset}
  Let \(f\) be a matrix of polynomials function. The \textbf{n-Zero Set} of \(f\) is
  \[
    \mathscr{V}_n(f) = \{X \in M_n (\CC) \mid f(X) =0\}.
  \]
  The \textbf{Zero Set} of \(f\) is
  \[
    \mathscr{V}(f) = \bigcup_{n \in \NN } \mathscr{V}_n(f).
  \]
\end{definition}

While the singular set encodes the matrices for which \(f(X)\) has a nontrivial
kernel, the directional singular set bundles this information together with the
kernel itself. Section 6 of Helton's \emph{Free Convex Algebraic Geometry}
\cite{heltonFree2013} shows how this can is analogous to the tangent plane of a
classical variety.  While it may seem counter intuitive to use a script ``Z''
for the singular set instead of the zero set, the singular set of a free
function is (in many cases) a more natural generalization of varieties.

{\color{fgreen} Lots of work has been done in the past decade} generating Null- and
Positivstellensatz for these three sets. In particular,
\cite{heltonFactorization2019} treats singular and zero sets while
\cite{heltonStrong2007} treats the directional zero set.

\section{Principal Divisors}%
\label{sec:prindiv}

Recall that given a differentiable traical free function \(f\), the free
gradient, \(\nabla f\) is the unique free functions satisfying
\[
  \tr (H \cdot \nabla f) = Df(X)[H]
\]
for all directions \(H\). On the other hand, for every square
\footnote{Meaning the output of \(g\) is a square matrix.}
\emph{free} function, \(g\), we can associate a determinantal function
\(\det g\)---which is defined in the obvious way. If \(f\) is a nontrivial
\emph{determinantal} function, then there is an induced tracial function,
\(\log f\) wherever \(f\) is nonzero.

% I forgot that determintal functions just spit out complex numbers so this
% doesn't matter...
%
%The presence of the logarithm may seem as we are outside the functional
%calculus of \cref{sec:ExtMuliVarFun}, which required self-adjoint matrices. For
%the values for which \(f(X)\) is diagonlizable, we can evaluate \(\log f(X)\)
%with the usual functional calculus. For an \(f(X)\) which is nonsingular
%and nondiagonalizable, recall that \(\HH^g\) is dense in \(\MM^{g} \) in the
%fine topology (and therefore any admissable topology). Since
%\(\log f(X): D \setminus \{X \mid f(X) \neq 0\} \)

\begin{definition}[Principal Divisors]%
\label{def:princdiv}
Let \(f\) be a nonzero \emph{determinantal} free function. Then the
\textbf{principal divisor} of \(f\) is
\[
  \divv f = \nabla \log f.
\]
Alternatively, if \(g\) is square \emph{free} function, then the principal divisor of
\(g\) is
\[
  \divv g = \nabla \log \det g
\]
\end{definition}

Before exploring the properties of \(\divv f\), it is worth acknowledging that
the notation is overloaded. Unfortunately, the principal divisors of both free
and determinantal functions have significant utility. One has to be careful
whether theorems concern the divisors of free functions or determinantal ones.
In light of this, the author has elected to italicize ``free'' and
``determinantal'' for the remainder of this section whenever there could be
ambiguity should one not read too carefully.

While it is trivial to verify, (simply use the properties of \(\log \) and the
linearity of \(\nabla\)) {\color{fgreen}} observe that
\[
  \divv fg = \divv f + \divv g.
\]

\begin{lemma}%
\label{lem:ob21}
  Let \(f,g\) be \(C^1\) nonzero \emph{determinantal} free functions. Then,
  \begin{enumerate}
    \item There exists an inverible locally constant determinantal functions
          \(c\) such that \(f=cg\) if and only if \(\divv f = \divv g\).
    \item \(\frac{f}{g} \)  has a \(C^1\) extension to the whole domain if and
          only if there is a \(C^1\) determinantal function \(h\) on the whole
          domain such that \(\divv f - \divv g = \divv h\).
    \item \(\frac{f}{g}\) and \(\frac{g}{f}\) have a \(C^1\) extension to the
          whole domain if and only if \(\divv f - \divv g\) has a continuous
          extension to the whole domain.
  \end{enumerate}
\end{lemma}
