%************************************************
\chapter{Zero Sets and Principle Divisors}\label{ch:ZeroDiv}
%************************************************

\section{Varieties, Classical and Free}%
\label{sec:varieties}

In the classical case, varieties are fairly easily to classify. Given some
(commutative) polynomial, \(f \in \CC [x_1, \dots, x_g]\) we define the zero set
\[
  V(f) = \{a \in \mathbb{A}^n \mid  f(a) =0\},
\]
where \(\mathbb{A}^n\) is complex affine \(n\)-space.  Varieties (both affine and
projective) are well studied in algebraic geometry
(Hartshorne's \emph{Alegbraic Geometry} \cite{hartshorneAlgebraic2008} is a
standard introduction). Of particular interest is a geometric invariant of a
variety called a \emph{divisor}. While divisors require robust machinery to
construct formally\footnote{Schemes, in particular.} one can think of them
(loosely) as formal sums of codimension one subvarieties. The concept of a
divisor lift naturally to the noncommutative setting, although varieties are
a touch more complex.

Let \(f\) be a matrix of polynomials on \(\MM^{g} \).
Unlike the classical case, it is not immediate what should be meant by
\(f(X)=0\)---is it enough for \(f(X)\) to be singular, or should \(f(X)\) be the
zero matrix? In light of this ambiguity, we make three definitions.
\begin{definition}[Singular Set]%
\label{def:singularset}
  Let \(f\) be a matrix of polynomials function. The \textbf{n-Singular Set} of \(f\) is
  \[
    \mathscr{Z}_n(f) = \{X \in M_n (\CC) \mid \det f(X) =0\}.
  \]
  The \textbf{Singular Set} of \(f\) is
  \[
    \mathscr{Z}(f) = \bigcup_{n \in \NN } \mathscr{Z}_n(f).
  \]
  Associated with the singular set is the \textbf{Directional Singular Set}:
  \[
    \mathscr{Z}_{\textrm{dir}}(f) = \{(X,v) \mid f(X)v = 0\}.
  \]
\end{definition}

\begin{definition}[Zero Set]%
\label{def:zeroset}
  Let \(f\) be a matrix of polynomials function. The \textbf{n-Zero Set} of \(f\) is
  \[
    \mathscr{V}_n(f) = \{X \in M_n (\CC) \mid f(X) =0\}.
  \]
  The \textbf{Zero Set} of \(f\) is
  \[
    \mathscr{V}(f) = \bigcup_{n \in \NN } \mathscr{V}_n(f).
  \]
\end{definition}

While the singular set encodes the matrices for which \(f(X)\) has a nontrivial
kernel, the directional singular set bundles this information together with the
kernel itself. Section 6 of Helton's \emph{Free Convex Algebraic Geometry}
\cite{heltonFree2013} shows how this can is analogous to the tangent plane of a
classical variety. While it may seem counter intuitive to use a script ``Z''
for the singular set instead of the zero set, the singular set of a free
function is (in many cases) a more natural generalization of varieties.
One needs to be careful when interfacing with the literature as these
definitions (including which of these three sets it the ``zero set'') are not
universal and each author seems to make their own choices.

{\color{fgreen} Lots of work has been done in the past decade} generating Null- and
Positivstellensatz for these three sets. In particular,
\cite{heltonFactorization2019} treats singular and zero sets while
\cite{heltonStrong2007} treats the directional zero set.

\section{Principal Divisors}%
\label{sec:prindiv}

Recall that given a differentiable traical free function \(f\), the free
gradient, \(\nabla f\) is the unique free functions satisfying
\[
  \tr (H \cdot \nabla f) = Df(X)[H]
\]
for all directions \(H\). On the other hand, for every square
\footnote{Meaning the output of \(g\) is a square matrix.}
\emph{free} function, \(g\), we can associate a determinantal function
\(\det g\)---which is defined in the obvious way. If \(f\) is a nontrivial
\emph{determinantal} function, then there is an induced tracial function,
\(\log f\) wherever \(f\) is nonzero.


\begin{definition}[Principal Divisors]%
\label{def:princdiv}
Let \(f\) be a nonzero \emph{determinantal} free function. Then the
\textbf{principal divisor} of \(f\) is
\[
  \divv f = \nabla \log f.
\]
Alternatively, if \(g\) is square \emph{free} function, then the principal divisor of
\(g\) is
\[
  \divv g = \nabla \log \det g
\]
\end{definition}

Before exploring the properties of \(\divv f\), it is worth acknowledging that
the notation is overloaded. Unfortunately, the principal divisors of both free
and determinantal functions have significant utility. One has to be careful
whether theorems concern the divisors of free functions or determinantal ones.
In light of this, the author has elected to italicize ``free'' and
``determinantal'' for the remainder of this section whenever there could be
ambiguity should one not read too carefully.

While it is trivial to verify, (simply use the properties of \(\log \) and the
linearity of \(\nabla\)), {\color{fgreen} observe that}
\[
  \divv fg = \divv f + \divv g.
\]
We will use this fact {\color{fgreen} to give some characterzations of divisors}

\begin{lemma}%
\label{lem:ob21}
  Let \(f,g\) be \(C^1\) nonzero \emph{determinantal} free functions. Then,
  \begin{enumerate}
    \item There exists an inverible locally constant determinantal functions
          \(c\) such that \(f=cg\) if and only if \(\divv f = \divv g\).
    \item \(\frac{f}{g} \)  has a \(C^1\) extension to the whole domain if and
          only if there is a \(C^1\) determinantal function \(h\) on the whole
          domain such that \(\divv f - \divv g = \divv h\).
    \item \(\frac{f}{g}\) and \(\frac{g}{f}\) have a \(C^1\) extension to the
          whole domain if and only if \(\divv f - \divv g\) has a continuous
          extension to the whole domain.
  \end{enumerate}
\end{lemma}

\begin{proof} \phantom{hello world}
\begin{enumerate}
  \item Suppose such a \(c\) existed. Then
        \[
          \divv f = \divv cg = \divv c + \divv g.
        \]
        But because \(c\) is locally constant, the presence of \(\nabla\) makes
        \(\divv c =0\), so \(\divv f = \divv g\).

        Conversely, suppose \(\divv f = \divv g\). But then
        \begin{align*}
          0 &= \divv f - \divv g\\
            &= \nabla \left( \log f - \log g \right) \\
          &= \nabla\log \frac{f}{g}.
        \end{align*}
        And so \(\log  \frac{f}{g}\) is locally constant! It follows that
        \(\frac{f}{g} \) is locally constant and hence we can write \(g=cf\) for
        some locally constant functions \(c\).

  \item Suppose there is a function \(h\) on the whole domain such that
        \(\divv h = \divv f - \divv g\)---then by part 1, \(h\) differs from
        \(\frac{f}{g}\) by a constant but is definied on the entire domain. It
        is immediate, then, that \(\frac{f}{g}\) extends to the whole domain.

        Conversely, suppose \(h\) is the continuous extension to the entire
        domain. But then
        \begin{align*}
          \frac{f}{g} &= h \\
               & \Downarrow \\
          \log f - \log g &= \log h \\
               & \Downarrow \\
          \divv f - \divv g &= \divv h.
        \end{align*}
  \item Part 3 follows immediately from part 2.
\end{enumerate}
\end{proof}

\begin{example}
  Consider the \emph{free} functions \(f(X,Y)=e^Xe^Y, g(X,Y)=e^{X+Y}\). In
  significant contrast to the classical case, \(X\) and \(Y\) do not commute,
  so \(f\neq g\). Before we look at the divisors of \(f\) and \(g\) it is
  pertinant to consider how \(f,g\) are actually defined---\(f\) and \(g\) are
  free functions defined on all of \(\MM^2\), so we are we are outside the functional
  calculus of \cref{sec:ExtMuliVarFun}, which required self-adjoint matrices.
  For the values for which \(X, Y\) are diagonlizable, we can evaluate \(f(X,Y)\)
  with the usual functional calculus. For an \(X\) or \(Y\) which is
  and nondiagonalizable, recall that \(\HH_n^2\) is dense in \(M_n(\CC)^2\) so
  we have level-wise continuous extension of \(f\) (and of course \(g\)) to all
  of \(\MM^2\).

  Now we consider the divisors of \(f\) and \(g\). Since they are free
  functions, recall that \(\divv\) is actually \(\divv \det\). But then,
  \begin{equation*}
  \begin{split}
    \divv e^Xe^Y &= \nabla \log \det \left( e^Xe^Y \right) \\
    &= \nabla \log \left( e^{\tr X} e^{\tr Y} \right) \\
    &= \nabla \left( \log e^{\tr X} + \log e^{\tr Y} \right) \\
    &= \nabla \tr X + \nabla\tr Y
  \end{split}
  \begin{split}
    \divv e^{X+Y} &= \nabla \log \det \left( e^{X+Y} \right) \\
    &= \nabla \log \left( e^{\tr (X+Y)}  \right) \\
    &= \nabla \tr (X+Y) \\
    &= \nabla \tr X + \nabla\tr Y
  \end{split}
  \end{equation*}
  And so we see that \(\divv e^Xe^Y = \divv e^{X+Y}\).
\end{example}

This example relies on the fact that \(\log \) plays nicely with \(e^{\tr X}\)
and one might wonder if there is an easier way to compute principal divisors.

\begin{theorem}
  Let \(f: D \to \MM^{\hat{d} \times \hat{d}}\) be a \(C^1\) free
  function\footnote{Since the codomain is \(\MM^{\hat{d}\times \hat{d}}\), one
    can view \(f\) as a \(\hat{d} \times \hat{d}\) matrix of free functions.}
  such that \(\det f \not\equiv 0\). Then
  \[
    \tr \left( H \cdot \divv f \right) = \tr \left(Df(X)[H] f(X)^{-1}\right)
  \]
\end{theorem}

\begin{proof}
  We begin by recalling Jacobi's formula, which gives us a way to understand the
  directional derivative of the determant in terms of the adjugate\footnote{The
    transpose of the cofactor matrix.}
  of a matrix. For a matrix \(X\),
  \[
    D \det X [H] = \tr \left( H \adj A \right).
  \]
  It will be imperative later in the proof to recall the following property of
  the adjugate. For an inevitable matrix \(X\),
  \[
    \adj (X) = \det(X) X ^{-1}.
  \]
  With these preliminaries sorted, we continue with the proof.
  Unraveling the definitions given above, the principal divisor of \(f\) (a
  \emph{free function}) is the unique free function on its nonsigular set satisfying
  \[
    D \log \det f (X)[H] = \tr (H \cdot \divv f).
  \]
  We compute
  \begin{align*}
    D \log \det f(X)[H] &=\left. \frac{d}{dt} \left[\log \det f(X+tH) \right]\right|_{t=0} \\
                        &= \frac{1}{\det f(X)} \left(\left. \frac{d}{dt} \left[ \det f(X+tH)\right]\right|_{t=0}\right)\\
                        &= \frac{1}{\det f(X)} \tr\left(\left. \frac{d}{dt} \left[ f(X+tH)\right] \adj f(X+tH)\right|_{t=0}\right)\\
                        &= \frac{1}{\det f(X)} \tr\left(Df(X)[H] \adj f(X)\right)\\
                        &= \tr \left( Df(X)[H] \frac{\adj f(X)}{\det f(X)}\right) \\
                        &= \tr \left( Df(X)[H] f ^{-1}(X) \right)
  \end{align*}
\end{proof}

The next section will treat divisors of polynomial and rational functions in
detail. Before continuing, we give one more example.

\begin{example}
  Let \(f(X,Y) = 1+XY\) and \(g(X,Y) =1+YX\). Using the previous theorem, we
  have that
  \begin{align*}
    \tr \left( (H_1,H_2) \cdot \divv f \right)
      &= \tr \left(Df(X,Y)[H_1,H_2] f(X,Y) ^{-1}\right)\\
      &= \tr \left( (H_1Y + X H_2) (1+XY) ^{-1}\right) \\
      &= \tr \left( H_1Y(1+XY)^{-1} + H_2(1+XY)^{-1}X \right)
  \end{align*}
  Appealing to trace duality (\cref{thm:trdual}), we see that
  \[
    \divv f = \left( Y(1+XY)^{-1},(1+XY)^{-1}X \right).
  \]
  With a nearly identical computation, we recover the principal divisor of \(g\)
  as well:
  \[
    \divv g  = \left( (1+YX)^{-1}Y,X(1+YX)^{-1} \right).
  \]
  Since \(Y(1+XY)=(1+XY)Y\), it follows that
  \(Y(1+XY)^{-1}= (1+XY) ^{-1}Y\), and so \(\divv f = \divv g\)!
\end{example}

\section{The Group of Divisors}%
\label{sec:divpoly}

For the remainder of this chapter, we will concern ourselves with the divisors
of square matrices of nc polynomials and nc rational functions. These are all
\(f\) free functions, so \(\divv\) will denote \(\divv \det\). We begin with a
theorem.

\begin{theorem}
  Let \(f,g\) be square matrices of nc polynomials such that
  \(\det f, \det g \not\equiv 0\). If \(\frac{\det f}{\det g}\) and
  \(\frac{\det g}{\det f}\) are entire, then \(\divv f = \divv g\).
\end{theorem}

\begin{proof}
  Consider \(\frac{\det f}{\det g}\) and \(\frac{\det g}{\det f}\) as functions
  \(M_n(\CC)\to \CC \). Since both of these are entire, \(\det f, \det g\) are
  both never 0---hence any zeroes or poles that they possess must be at
  infinity. Suppose that \(\frac{\det f}{\det g}\) is unbounded. Depending on how
  the degrees of \(\det f\) and \(\det g\) compare, there is either a zero or a
  pole at infinity. But this means that \(\frac{\det g}{\det f}\) has either a
  zero or a pole at \(0\). Either way we have a contradiction, and so
  \(\frac{\det f}{\det g}\) is bounded (and entire)---hence constant.

  We now appeal to lemma \ref{lem:ob21}, part 3. Since \(\frac{\det f}{\det g}\)
  and its reciprocal both have \(C^1\) extension (namely themselves), we have a
  levelwise constant function \(h\) such that \(\divv f - \divv g = \divv h\).
  But clearly \(\divv h\) is 0, so \(\divv f = \divv g\)!
\end{proof}

One of the major themes of the development of principal divisor of free
functions (like in \cite{pascoeFreeNoncommutativePrincipal2020}) is that much of
the structure of divisors is an immediate corollary of the structure of
\(\det f\). For example, the following theorem is proven almost entirely by its
lemma.

\begin{theorem}
  Let \(r\) be a nondenerate square matrix of nc rational expressions, such that
  \(\det X \not\equiv 0 \). Then there exists square matrices of nc polynomials
  such that
  \[
    \divv r = \divv p - \divv q
  \]
\end{theorem}

\begin{proof} We begin with a lemma.
  \begin{lemma}
    Let \(r\) be a nondenerate square matrix of nc rational expressions, such that
    \(\det X \not\equiv 0 \). Then there exists square matrices of nc polynomials
    such that
    \[
      \det r = \frac{\det p}{\det q} = \det (pq ^{-1})
    \]
  \end{lemma}

  \begin{proof}
    Recalling \cref{thm:ratchar}, let \(r = b^* L ^{-1} c\). We claim that
    \[
      p = \begin{bmatrix} L&c\\-b&0 \end{bmatrix}  \qquad \textrm{ and } \qquad q = L.
    \]
    We see that
    \begin{align*}
      \left.\det \begin{bmatrix} L&c\\-b&0 \end{bmatrix}  \right / \det L
        &=\det \begin{bmatrix} L&c\\-b&0 \end{bmatrix} \det \begin{bmatrix} L ^{-1}&0\\0&1 \end{bmatrix} \\
        &= \det \begin{bmatrix} 1&c\\-bL ^{-1}&0 \end{bmatrix} .
    \end{align*}
    Now we recall the formula for the determinant of a block matrix:
    \[
      \det \begin{bmatrix} A&B\\C&D \end{bmatrix}  = \det A\det D - CA^{-1}B.
    \]
    With this in hand, we see that
    \(\det \left[ \begin{smallmatrix}1&c\\-bL^{-1}&0\end{smallmatrix} \right] = \det b^*L ^{-1}c\),
    and we are done.
  \end{proof}
  Now take the \(\divv\) of both sides of the lemma to get the required result.
\end{proof}

Just as in the classical case, these is a deep link between factorization of
polynomials, subvarieties, and principal divisors. Before we can explore this
link in the noncommutative setting, we need a definition.

\begin{definition}[Atomic]
% \label{def:label}
  A square matrix of polynomials \(p\) is \textbf{atomic} if
  \(\det p \not\equiv 0\) and if \(p_1p_2=p\), then either \(\det p_1\) or
  \(\det p_2\) is locally constant.
\end{definition}

Atomic square matrices of polynomials function like irrecudible factors of
tradition (commutative) polynomials. While we cannot have truly ``unique''
factorization, we do have factorization into atoms. In
\cite{heltonFactorization2019}, Helton et al.\ prove the following theorem:

\begin{theorem}
  Let \(f\) is a square matrix of nc polynomials and \(p\) an atom. If we
  \(f = p_1 \cdots p_k\) is the factorzation of \(f\) into atoms, then
  \[
    \mathscr{Z}(p) \subset \mathscr{Z}(f)
  \]
  if and only if \(\det p = \det (c p_i)\) for one of the atoms of \(f\) and
  \(c\) a nonzero constant.
\end{theorem}

With the help of lemma \ref{lem:ob21} (part 1) this says that factorization is
unique up to equivalence of principal divisors.
