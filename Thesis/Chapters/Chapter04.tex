\chapter{Monodromy, Global Germs, Algebraic Topology}\label{ch:monodromy}

The results of the last three chapters seem hopeful---free analysis seems to be
able to generalize many classical results, as listed in \cref{sec:freeanal}. As
previously mentioned, the free topology admits an Oka-Weil-type theorem.
While this is promising, the only compact sets in the free topology are the
envelopes of a finite collection of points.

It is the opinion of the author that all of these topologies (fine, fat, free,
etc.) are definitively
broken. As shown above, the free topology lacks a wealth of compact sets.
The fine topology (and therefore any admissible topology)
fails to be \(T_1\), let alone Hausdorff---notice that any open set containing
\(X\) must also contain \(X\oplus X\). Further, given any free function \(f\)
on an nc-domain \(D\), if \(f\) is locally bounded on each \(D_n\) then \(f\)
is analytic (admits a power series representation). There are two ways so view
this result: First, one can accept that analytic functions are a dime a dozen on
\(\MM^{g} \). Alternatively, one can be skeptical that the topological
structures put on \(\MM^{g} \) are indeed the natural choice. The work of J.E.
Pascoe in \cite{pascoeFreeNoncommutativePrincipal2020} seeks to solve some of
these issues by extending some of the concepts of traditional algebraic
topology.
\begin{figure}[h!]%
\centering
  \def\svgwidth{0.8\columnwidth}
  \import{./Chapters/img/part 2/}{monodromycurve.pdf_tex}
\caption{Analytic continuation along a curve}
\label{fig:monocurve}
\end{figure}

\newpage

\section{Classical Monodromy}%
\label{sec:classmono}

In the study of functions of a single complex variable,\footnote{There are nearly
  identical ``monodromy'' theorems for functions of one and several complex
  variables, but we will only treat functions of a single variable in this section.}
many of the central
theorems surround the idea of analytic continuation. Given some analytic
function \(f\) on a domain \(\Omega \subset\CC \) and a larger domain
\(\overline{\Omega} \supset \Omega\), we can (with sufficient
``niceness'' conditions) extend \(f\) to an analytic function \(g\) on
\(\overline{\Omega}\). In particular, given some path \(\gamma\) which start in
\(\Omega\) we can analytically continue \(f\) \emph{along} \(\gamma\) by recomputing the
power series on overlapping disks with their centers on \(\gamma\). The
standard picture of this process is \cref{fig:monocurve}.

Our path \(\gamma\) must avoid any potential poles of \(f\) so that we may
compute the power series, but the uniqueness of such an extension is not
obvious. This is where the aforementioned niceness conditions come into play!
For example, consider the setup of \cref{fig:monolog}:

\begin{figure}[h!]%
\centering
  \def\svgwidth{0.52\columnwidth}
  \import{./Chapters/img/part 2/}{monodromylog.pdf_tex}
\caption{Two paths in \(\CC \)}
\label{fig:monolog}
\end{figure}

\begin{example}%
\label{ex:logmono}
If we let \(f(x)=\Log x\) be the principle branch of the complex logarithm the
defined on the right half plane, and continue \(f\) along \(\gamma_1\) and
\(\gamma_2\) we get two functions \(f_1\) and \(f_2\) which are analytic at
\(\beta\), but they don't agree! In this case, \(f_1(\beta)\) and \(f_2(\beta)\)
disagree by exactly \(2\pi i\).
\end{example}

The monodromy theorem gives sufficient conditions for the continuation along
two curves to agree:

\begin{theorem}[Monodromy I]%
\label{thm:mon1}
  Let \(\gamma_1, \gamma_2\) be two paths from \(\alpha\) to \(\beta\) and
  \(\Gamma_s\) be a fixed-endpoint homotopy between them. If \(f\) can be
  continued along \(\Gamma_s\) for all \(s \in [0,1]\), then the continuations
  along \(\gamma_1\) and \(\gamma_2\) agree at \(\beta\).
\end{theorem}

In the example above, any homotopy between the two paths must pass through the
origin---where \(\Log x\) fails to be analytic---and hence the two continuations
disagree at \(\beta\).
An equivalent formulation of the monodromy theorem concerns extending a
functions to a larger domain:
\clearpage
\begin{theorem}[Monodromy II]%
\label{thm:mon2}
  Let \(U \subset \CC \) be a disk in \(\CC \) centered at \(z_0\) and
  \(f: U \to \CC \) an analytic function. If \(W\) is
  an open, simply connected set containing \(U\) and \(f\) continues along any
  path \(\gamma \subset W\) starting at \(z_0\), then \(f\) has a unique
  extension to all of \(W\).
\end{theorem}

This second formulation gives another perspective on \(\Log x\). In the example,
\(U\) is a disk around \(\alpha\) that stays in the right half plane and \(W\)
is \(\CC \setminus \{0\} \). While \(\Log x\) continues along any path in
\(\CC  \setminus \{0\} \), the larger domain is \emph{not} simply connected, so
monodromy fails.


\section{Free Monodromy}%
\label{sec:freemono}

There is an analogous theorem to \cref{thm:mon1,thm:mon2} in the free setting---
initial proven by J.E.\ Pasocoe in \cite{pascoeNoncommutative2020}.
In the classic case, the larger set \(W\) must be simply connected. In the free
setting, however, the theorem is much more powerful. Before we state and prove
the theorem, recall that free functions respect direct sums---so if
\(f: D\to \MM^{\tilde{g}} \) is a free function,
\[
  f(X \oplus Y ) = f(X) \oplus f(Y).
\]
Given two paths \(\gamma_1, \gamma_2 \in D_n\), we can take their direct sum in
the obvious way
\[
  (\gamma_1\oplus\gamma_2 )(t) = \gamma(t) := \begin{bmatrix} \gamma_1(t) & \\& \gamma_2(t) \end{bmatrix}
\]
to obtain a path in \(D_{2n}\). If \(f\) is a free function defined on
\(B \subset D\), and then we can analytically continue \(f\) along \(\gamma\)
(presuming that \(\gamma\) originates in \(B\)). If \(F\) is the resulting
function defined at \(\gamma(1)\), and \(F_1,F_2\) are the continuations at
\(\gamma_1(1), \gamma_2(1)\) respectively, then a routine computation shows that
\[
  F(\gamma(1)) = \begin{bmatrix} F_1(\gamma_1(1)) & \\& F_2(\gamma_2(1))\end{bmatrix} .
\]
With this preliminary result, we can introduce Universal Monodromy.
\begin{theorem}[Free Universal Monodromy]%
  \label{thm:freemono}
  Let \(f\) be an analytic free function defined on some ball \(B \subset D\),
  for \(D\) an open, connected free set.
  Then \(f\) analytically continues along every path in \(D\) if and only if
  \(f\) has a unique analytic continuation to all of \(D\).
\end{theorem}

\begin{proof}
  [Proof (from \cite{bickelPascoe2021})]
  The fact that a unique extension to all of \(D\) implies that \(f\) has a
  continuation along any \(\gamma\) is immediate.

  Now suppose that \(f\), a free function, analytically continues along every
  path in \(D\). Fix \(X \in B_n\) and pick some and let \(\gamma_1,\gamma_2\)
  be two paths taking \(X\) to some \(Y \in D_n\). Let \(F_1,F_2\) be the
  analytic continuation of \(f\) along \(\gamma_1,\gamma_2\) respectively. We
  seek to show that
  \(F_1\) and \(F_2\) agree in some neighborhood of \(\gamma_1(1)\)! Let \(\hat{\gamma}, \gamma\) be paths in
  \(D_{2n}\) defined by
  \[
    \hat{\gamma} = \begin{bmatrix} \gamma_1&\\&\gamma_2 \end{bmatrix} \qquad
    \gamma = \begin{bmatrix} \gamma_2&\\&\gamma_1 \end{bmatrix}.
  \]
  We have a homotopy between \(\hat{\gamma}\) and \(\gamma\) given by
  \[
    \Gamma(t,s) =
    \begin{bmatrix} \cos (s\frac{\pi}{2}) & \sin (s\frac{\pi}{2})\\-\sin (s\frac{\pi}{2}) & \cos (s\frac{\pi}{2}) \end{bmatrix}
    \begin{bmatrix} \gamma_1(t)&\\&\gamma_2(t) \end{bmatrix}
    \begin{bmatrix} \cos (s\frac{\pi}{2}) & -\sin (s\frac{\pi}{2})\\\sin (s\frac{\pi}{2}) & \cos (s\frac{\pi}{2}) \end{bmatrix}.
  \]
  Indeed, one easily checks that
  \[
    \Gamma(t,0) = \hat{\gamma}\qquad
    \Gamma(t,1) = \gamma \quad \quad
    \Gamma(0,s) = X\oplus X\qquad
    \Gamma(1,s) = Y \oplus Y
  \]
  But since \(\hat{\gamma}\) and \(\gamma\) are homotopic we can apply the
  classical (albeit multivariable) monodromy theorem---so we know that the
  analytic continuations of \(f\) along \(\hat{\gamma},\gamma\) must agree near
  \(Y \oplus Y\). Since free functions respect direct sums, if we let
  \(\hat{F}\) and \(F\) denote the continuations of \(f\) along
  \(\hat{\gamma},\gamma\) respectively, we obtain the
  following chain of equalities:
  \[
    \begin{bmatrix} F_1(\gamma_1) &\\&F_2(\gamma_2) \end{bmatrix}  =
    \hat{F}(\gamma_1 \oplus \gamma_2) = F (\gamma_2 \oplus \gamma_1) =
    \begin{bmatrix} F_2(\gamma_2)& \\ & F_1(\gamma_1) \end{bmatrix}
  \]
  In particular, we see that \(F_1(\gamma_1)=F_2(\gamma_2)\)---so \(F_1\) and
  \(F_2\) agree!
\end{proof}

In the free case, the ``larger'' set need not be simply connected. Analytic
continuations of free functions, then, cannot be used to detect holes in matrix
domains. It will turn out, however, that the tracial and determinental functions
introduced in \cref{sec:TracGrad} can detect holes and produce an analogue of
the fundamental group!

\section{The Germ of Function}%
\label{sec:germs}

As studied in complex analytic and measure theoretic settings, if our space is sufficiently
structured functions are defined by their local behavior. This idea can
be generalized to arbitrary topological spaces with a construction from
algebraic geometry.

Let \(X\) be a topological space. To any open set \(U\) we associate \(C(U)\),
the ring of continuous functions \(f: U \to \RR \) (where addition and
multiplication are defined point-wise).
Given any \(V \subset U\), notice that for any continuous function \(f\) on \(U\), we
can restrict \(f\) to \(V\) and maintain continuity. This gives two maps:

\begin{equation*}
\begin{split}
	V &\lhook\joinrel\longrightarrow U \\
  v &\longmapsto v
\end{split} \qquad \qquad \qquad
\begin{split}
	C(U) &\lhook\joinrel\longrightarrow C(V) \\
  f &\longmapsto f \mid _V
\end{split}
\end{equation*}

Notice that the induced function goes the ``other way.'' This construction is an
example of a sheaf of rings\footnote{To be completely rigorous, a sheaf needs
  additional axioms, but the sheaf of continuous functions is one of the
  prototypical examples so the full defition is not needed in this
  context.}---since \(C(U)\) has a ring structure. We can similarly define sheaves
of abelian groups or sets: to each open set in \(X\) we assign a group (or set)
such that there are analogous restriction maps. For our purposes, these will
always be groups/sets of functions and the restriction maps are the natural ones.

We are interested in the general behavior of continuous functions at some
\(x \in X\). Define \(\mathfrak{C}_x\) to be the set of all functions defined on a
neighborhood of \(x\):
\[
  \mathfrak{C}_x = \{f \in C(U) \mid x \in U \subset X \text{ is open}\}.
\]
By convention, we refer to elements of \(\mathfrak{C}_x\) as a pair, \((f,U)\) of
a continuous function and the open set on which it is defined.
In light of the inclusion maps given above, it obvious that \(\mathfrak{C}_x\)
will have ``duplicate'' elements. Therefore, we define an equivalence relation
on \(\mathfrak{C}_x\) by
\[
(f,U) \sim (g,V) \Leftrightarrow \textrm{ there exists }
W \subset U \cap V \textrm{ where } f \mid _W = g\mid _W.
\]
In a sheaf-theoretic
context, \(\faktor{\mathfrak{C}_x}{\sim} \) is called the \textbf{stalk} at \(x\)
and elements of the stalk are \textbf{germs} at \(x\). If we are dealing with
sheaves of groups or sets, this construction remains unchanged---\ie{} can still
define the stalk at given point. While it will not come into play, it is worth
noting that the stalk inherits the algebraic structure of the original
sheaf---\eg{} for a sheaf of rings, the stalk has a natural ring
structure.

Sheafs of rings/groups/sets of functions arise naturally in many areas of
mathematics. For example, if \(X\) happens to be a smooth manifold, we may
replace \(C(U)\) with \(C^\infty(U)\), the ring of smooth functions into
\(\RR \) and then obtain germs of smooth functions.  Similarly, if \(X\) is a
complex manifold we can construct germs of holomorphic functions.

\begin{example}
Consider, again, example \ref{ex:logmono}. Our function \(f(x)= \Log x\) has
a germ in \(\Omega\). In particular, both \(f_1\) and \(f_2\) belong to the
equivalence class \([(f,\Omega)]\) as all three functions agree on \(\Omega\).
From this, we see the aptness of the name germ: germs capture the
local behavior of function. Colloquially, this is the ``heart'' of a
function similar to the germ of seed.\footnote{Sheaf theory abounds with
  agrarian nomenclature.}
\end{example}


As usual, lifting this construction to the free context requires some nuance.
For \(U \subset D\) open, the set of tracial functions on \(U\) (denoted
\(C_{\text{tr}}(U)\)) does not form a ring---it is closed under addition but not
multiplication. Given two tracial functions, \(f,g \in C_{\text{tr}}(U)\), we see
that
\begin{align*}
  (f+g)(X \oplus Y) &= f(X \oplus Y) + g(X \oplus Y) \\
                    &= f(X) + f(Y)+g(X) + g(Y) \\
                    &= (f+g) (X) + (f+g)(Y)
\end{align*}
but,
\begin{align*}
  (fg)(X \oplus Y) &= f(X \oplus Y)g(X \oplus Y) \\
                   &= (f(X) + f(Y))(g(X) + g(Y)) \\
                   &= (fg) (X) + (fg)(Y) + f(X)g(Y)+f(Y)g(X).
\end{align*}

Thankfully, however, the construction remains unchanged if we substitute a ring
of functions for an abelian group of functions (with the identity being
\(f \equiv 0\) and inverses given by simply negating the output). In the case of
determinental and free functions (which play a lesser role in the theory to be
developed) there is not a natural algebraic structure for the corresponding
sheaves, so they are simply sheaves of sets.

\section{The Tracial Fundamental Group}%
\label{sec:trpi1}

While free monodromy means that free functions cannot detect the topology of
free sets, the same is not true for a general tracial
function! Following \cite{pascoeFreeNoncommutativePrincipal2020}, we will need
some definitions.

\begin{definition}[Anchored]%
  \label{def:anchored}
Let \(D \subset \MM^{g} \) be a connected, open, free set. If there exists a
nonempty, simply-connected, open, free \(B \subset D\), then we say that \(D\)
is \textbf{anchored}.
\end{definition}

\begin{definition}[Global Germ]
  \label{def:globgerm}
  For \(D\) an anchored set, and \(B \subset D\) its anchor, we call a tracial
  function \(f:B\to \CC \) a \textbf{global germ} if it analytically continues
  along every path in \(D\) which starts in \(B\).
\end{definition}

In order to define the fundamental group, we need a notion of a path in \(D\).
Traditionally, a path taking \(X\) to \(Y\) is a continuous function
\(\gamma: [0,1]\to D\) such that \(\gamma(0)=X\) and \(\gamma(1)=Y\).
Unfortunately, this disregards the fiber of \(X\) and \(Y\). An mentioned in
section \ref{sec:TopManUniv}, a proper topological theory should account for
identification of the fibers.

\begin{definition}[Essential Path]
  \label{def:esspath}
  A continuous function \(\gamma:[0,1]\to D\) \textbf{essentially takes} \(X\) to \(Y\) if
  \begin{align*}
    \gamma(0) = X^{\oplus \ell},& \;\text{ for some \(\ell \in \NN \)}\\
    \gamma(1) = Y^{\oplus k},& \;\text{ for some \(k \in \NN \)}.
  \end{align*}
\end{definition}

\begin{figure}[h!]
\centering
  \def\svgwidth{0.9\columnwidth}
  \import{./Chapters/img/part 2/}{essentialpath.pdf_tex}
\caption{A path essentially taking \(X\) to \(Y\)}
\label{fig:esspath}
\end{figure}
A path essentially taking \(X\) to \(Y\) is a path from some element of the
fiber of \(X\) to some element of the fiber of \(Y\). Just as in the classical
case, essential paths have a product. First, we need a way to take the direct sum
of paths.

\begin{definition}[Direct Sum of Paths]%\label{def:sumpath}
  Given \(\gamma\) essentially taking \(X\) to \(Y\) and \(\beta\) taking \(Z\)
  to \(W\), define
  \[
    \gamma\oplus\beta(t) := \begin{bmatrix} \gamma(t)&0\\0&\beta(t) \end{bmatrix}.
  \]
\end{definition}

It is not, in general, true that \(\gamma\oplus\beta\) essentially takes
\(X\oplus Z\) to \(Y \oplus W\). However, if \(\gamma\) essentially takes \(X\)
to \(Y\), then so does \(\gamma\oplus\gamma\). As with matrices, we define
\[
  \gamma^{\oplus k} := \underbrace{\gamma\oplus \cdots \oplus\gamma}_{k \text{ times}}
\]

With these preliminaries, we can now define a concatenation product for
essential paths:
\begin{definition}[Concatenation Product]%
\label{def:concatprod}
  Let \(\gamma\) and \(\beta\) be paths taking \(X\) to \(Y\) and \(Y\) to \(Z\)
  respectively. We define their product to be the path essentially taking \(X\)
  to \(Z\) given by
  \[
    \beta\gamma(t) :=
    \begin{cases}
      \gamma^{\oplus k}(2t) & t \in [0,0.5) \\
      \beta^{\oplus\ell} (2t-1)& t \in [0.5,1]
    \end{cases}
  \]
  where \(k\) and \(\ell\) are positive integers chosen to make
  \(\gamma^{\oplus k}\) and \(\beta^{\oplus \ell}\) like size matrices for each
  \(t \in [0,1]\).
\end{definition}

With essential paths and their product we can build the first analogue of the
fundamental group. Let \(D\) be an anchored space with \(B\) its anchor. For
\(X \in B\), we define \(\pi_1(D,X)\) to be the set of path essentially taking
\(X\) to \(X\) up to traditional homotopy equivalence and the relation
\(\gamma=\gamma^{\oplus k}\). Section 6 of
\cite{pascoeFreeNoncommutativePrincipal2020} explores this construction in
detail, including proving its commutativity.
At the moment, not much can be said about the full fundamental group. Instead,
we restrict ourselves to a different (albeit related) group of paths which are
determined by the analytic continuation of tracial functions.

Given a path essentially taking \(X\) to \(Y\) we can view the path as coupled
with its endpoint. For \(B\) and anchor and \(f\) a global germ, we can
reasonably define \(f(\gamma)\): analytically continue \(f\) along \(\gamma\) and define
\[
  f(\gamma) := \frac{1}{k} f(Y^{\oplus k}).
\]
Since we can evaluate paths with global germs, we can use global germs to
distinguish between certain paths.

\begin{definition}[Trace Equivalent]
\label{def:label}
  Let \(B \subset D\) be an anchor and fix \(X \in B\). If \(\gamma\) and
  \(\beta\) both essentially take \(X\) to \(Y\), we say they are
  \textbf{trace equivalent} if, for every global germ \(f\) and every path
  \(\delta\) taking \(Y\) to \(Z\), \(f(\delta\gamma)=f(\delta\beta)\).

  That is, trace equivalent paths are those which cannot be told apart via
  analytic continuation of global germ.
\end{definition}

Under trace equivalence, the normalization given above implies
\(\gamma = \gamma^{\oplus k}\) since both essentially take \(X\) to \(Y\).
Further, since homotopic paths have the same analytic continuation, homotopic
paths are trace equivalent. This allows us to define another fundamental group
which will be our central object of study.
\begin{definition}[Tracial Fundamental Group]%
\label{def:trpi1}
 Let \(D\) be an anchored space with \(B\) its anchor. For \(X \in B\) define
 \(\pi_1^{\textrm{tr}}(D,X)\) to be the group of trace equivalent paths
 essentially taking \(X\) to \(X\).
\end{definition}
If \(D\) is connected, then \(\pi_1^{\textrm{tr}}(D)\) is independent of our
choice of base point---in fact, the isomorphism from the classical case
works here as well. The identity is given by \(\gamma_X\), the
constant path at \(X\), and inverses given by
\[
  \gamma ^{-1}(t) = \gamma (1-t).
\]
Note that, since fixed endpoint homotopic paths are trace equivalent,
\(\pi_1^{\textrm{tr}}(D)\) is a quotient of \(\pi_1(D)\). We can construct a covering
space for \(D\) with respect to \(\pi_1^{\textrm{tr}}(D)\) similar to the
construction of the universal cover in \cite{hatcherAlgebraic2002}.

\begin{definition}[Tracial Covering Space]%
\label{def:trcover}
  For \(X \in B \subset D\), the \textbf{tracial covering space} of \(D\) is the
  set of paths {(up to tracial equivalence\footnote{From here on, unless otherwise specified, we will only refer to
    paths up to trace equivalence.})}
  in \(D\) starting at \(X\):
  \[
    C^{\textrm{tr}}(D) = \{[\gamma] \mid \gamma \text{ a path essentially taking
    \(X\) to \(Y\)}\}
  \]
\end{definition}

Since we identify paths with their terminal endpoint, we have the natural
covering space map \(\rho: C^{\textrm{tr}}(D) \to D, [\gamma]\mapsto Y\). In order
for this map to be continuous (and obey the rest of the axioms of a covering
space), we need to endow \(C^{\textrm{tr}}(D)\) with a topology. We do so be
defining a metric
\(d: C^{\textrm{tr}}(D) \times  C^{\textrm{tr}}(D) \to \RR \cup \{\infty\} \).
Let \(\gamma_1\) be a path essentially taking \(X\) to \(Y\) and \(\gamma_2\) a
path essentially taking \(X\) to \(Z\). If \(Y\) and \(Z\) are different sizes,
set \(d(\gamma_1,\gamma_2)=\infty\). On the other hand, say that both \(Y\) and
\(Z\) are \(n \times n \) matrices and let \(\Gamma_{Y,Z}\) be the set of path
in \(D_n\) taking \(Y\) to \(Z\). If \(\| \gamma \| \) denotes the length of
\(\gamma\) in \(\CC ^{n^2}\), then we set
\[
  d(\gamma_1,\gamma_2) := \inf \{\| \gamma \| \mid \gamma \in \Gamma_{Y,Z} \textrm{
  such that } \gamma\gamma_1=\gamma_2\}.
\]
With the topology induced by the metric, one can
easily verify that we do, indeed, have a covering space.

Because \(B\) is simply connected, for any \(Y \in B\) there is exactly one path
essentially taking \(X\) to \(Y\). In light of this, there is a natural
inclusion \(B \hookrightarrow C^{\textrm{tr}}(D)\). Given a global germ, \(f\),
we induce a function on the covering space (given by \(f(\gamma)\)), which the
norm on \(C^{\textrm{tr}}(D)\) forces to be analytic.

\section{A Bit of Cohomology}%
\label{sec:cohomo}

For a complete treatment of Cohomology, see Allan Hatcher's famous
\emph{Algebraic Topology} \cite{hatcherAlgebraic2002}. As a quick review, given
some chain complex
\[
  \cdots
  \xleftarrow{\partial_{n-1}}
  C_{n-1}
  \xleftarrow{\partial_{n}}
  C_{n}
  \xleftarrow{\partial_{n+1}}
  C_{n+1}
  \xleftarrow{\partial_{n+2}}
  \cdots
\]
we can form the for the \textbf{cochain complex} as follows: First, fix some
abelian group \(G\). The objects in our cochain complex are
\(C^\bullet = \Hom (C_\bullet, G)\), the group of morphisms \(C_\bullet \to G\).
The maps are induced ones
\[
  d = \partial^* : \Hom(C_n,G) \to \Hom(C_{n+1},G).
\]
Put together, this gives us a complex with the arrows reversed
\[
  \cdots
  \xrightarrow{d_{n-2}}
  C^{n-1}
  \xrightarrow{d_{n-1}}
  C^{n}
  \xrightarrow{d_{n}}
  C^{n+1}
  \xrightarrow{d_{n+1}}
  \cdots
\]
Computing the homology of this dual complex gives the \emph{cohomology}
groups, \(H^\bullet (C; G)= \textrm{Ker}\;  d / \textrm{Im}\; d\)---which only
depend on the homology of the original complex and the choice of \(G\). Of
particular interest is \textbf{De Rham Cohomology}, where \(C^k = \Omega^k\), the
set of \(k\)-forms on a manifold, and \(G=\RR \). In this case, the boundary map
is given by the exterior derivative. For a full construction of De Rham
Cohomology, see \cite{leeIntroduction2013}. In the classical case we say that a
\(k\)-form is \textbf{closed} if \(\text{d} f =0\) and \textbf{exact} if there
exists a \(k\)--1-form, \(g\), such that \(\text{d} g=f\). The \(k\)th De Rham
cohomology group, then, is the vector space of closed forms moduluo the exact
forms.

A full cohomology theory has yet to be developed in the free setting. Lifting
De Rham cohomology appears promising given that \(\MM^{g} \) carries
a natural (if complex) manifold structure. While there is not a full
generalization of the exterior derivative, recall that for any tracial function \(f\),
we have that \(\nabla f\) is a free function. If \(\mathcal{T}\) is the set of
tracial functions and \(\mathcal{F}\) is the set of free functions, we have the
beginning of a cochain complex
\[
  0 \rightarrow \mathcal{T} \xrightarrow{\nabla} \mathcal{F} \rightarrow \cdots.
\]
While we cannot define ``closed'' and ``exact'' for general functions on
\(\MM^{g} \) (in part because we don't know what the general cochain groups are)
we can define them on \(\mathcal{F}\).
\begin{definition}[Exact]%
\label{def:exact}
  A free function \(g: D \to \MM^{g} \) is \textbf{exact} if there exists a
  tracial function \(f: D \to \CC \)
  such that \(\nabla f = g\).
\end{definition}
\begin{definition}[Closed]%
\label{def:closed}
  A free function \(g: D \to \MM^{g} \) is \textbf{closed} if
  \[
    \tr \left( K \cdot Dg(X)[H] \right) = \tr \left( H \cdot Dg(X)[K] \right)
  \]
  for all directions \(H,K\).
\end{definition}

While exactness is a direct lift of the classical condition, our definition of
closed is decidedly unenlightening. Recall from the discussion at the end of
\cref{sec:MatUniv} that the combination of the trace of a dot product is a
bilinear form on \(\MM^{g} \) which we can think of as an inner
product.\footnote{Albeit, without conjugates.}
Recall that, in the classical case, a function on \(\RR ^n\) is exact if the
Jacobian is symmetric.
If we imagine \(Dg(X)[H]\) as analogous to the Jacobian of a classical function
evaluated in the direction \(H\), and \(\tr (\;\; \cdot \;\;)\) as the
inner product, then the closed condition is exactly the same!
With our definitions, we can
define the first tracial cohomology group.

\begin{definition}[First Tracial Cohomology Group]%
\label{def:firsttrcohomo}
  The \textbf{first tracial cohomology group} is the vector space of closed free
  functions moduluo the exact free function. We write \(H^1_{\textrm{tr}}(D) \).
\end{definition}

At first glance, \(H^1_{\tr}(D)\) seems rather convoluted, arbitrary, and not
particularly useful.
\footnote{In all fairness, this is most people's reaction when the encounter
  cohomology for the first time.}
Thankfully, we can put the tracial cohomology group to immediate use in
understanding the structure of \(\pi_1^{\tr}(D)\). Recall that, by definition, a
global germ \(f:B\to \CC \) analytically continues along every path. It follows,
then, that \(\nabla f\) must analytically continue along every path as
well---simply analytically continue \(f\) along the path and then take its
gradient. Since \(\nabla f \) is a free function, universal monodromy
(\cref{thm:freemono}) tells us that \(\nabla f\) has a unique continuation to
all of \(D\).

At first glance it would seem that \(\nabla f\) (for \(f\) a
global germ) is trivial in \(H_{\tr}^1(D)\). This, however, is not the case. A
free function \(g\), is exact if there is a \emph{tracial function}, \(\hat{f}\)
defined on \emph{all of} \(D\) such that \(g= \nabla \hat{f}\). Given a
\emph{global germ} \(f: B \to \CC \), it is not always the case that \(f\) has a
unique extension to all of \(D\)---hence \(\nabla f \) is not necessarily exact!



Because monodromy holds for the gradient of a global germ we know that for any
\(\gamma \in \pi_{\tr}^1(D)\), \(f:B\to \CC \) a global germ, and \(\gamma'\) a
path starting our anchor point \(X\), the function
\[
  \Phi_\gamma^f (\gamma')= f(\gamma'\gamma)-f(\gamma')
\]
is locally constant as a function on \(C^{\tr}(D)\). Because of the metric we
put on \(C^{\tr}(D)\), showing that this function
is locally constant is rather finicky. This result has a direct analogue in
classical complex analysis and the proof similar so it will be omitted.
We remark that \(\Phi_\gamma^f\) measures the action of \(\gamma\) on a global
germ \(f\) when continuing along a path \(\gamma'\). We can use to prove the
following technical lemma.


\begin{lemma}%
\label{lm:technical}
  Let \(D\) be an nc domain. For any \(\alpha,\beta \in \pi_1^{\tr}(D)\) and
  global germ \(f\),
  \[
    f(\alpha\beta) - f(\alpha) = f(\beta) - f(\gamma_X).
  \]
\end{lemma}

\begin{proof}
  First, see that \(f(\beta)=f(\gamma_X\beta)\), so we need to show that
  \(\Phi_\beta^f(\alpha) = \Phi_\beta^f(\gamma_X)\). Let \(\Gamma\) be a path in
  \(C^{\tr}(D)\) defined by
  \[
    \Gamma(t) = \alpha\Big|_{[0,t]}.
  \]
  That is, \(\Gamma\) is the path (in the space of paths) where, at each time
  step \(\Gamma\) is the path given by going \(t\) through \(\alpha\).
  Continuity of \(\Gamma\) is immediate from our metric---since the distance
  between \(\Gamma(t)\) and \(\Gamma(t+\varepsilon)\) can be made arbitrarily
  small.

  Since we have path between \(\gamma_X\) and \(\alpha\), they are in the same
  path component of \(C^{\tr}(D)\). Therefore, since \(\Phi_\beta^f\) is locally
  constant, it must be the case that \(\Phi_\beta^f(\alpha) = \Phi_\beta^f(\gamma_X)\).
\end{proof}

Given \(f\) a global germ and \(X \in B_n\) the anchor point and
\(\gamma \in \pi_1^{\tr}(D)\), define
\[
  c^f(\gamma) := \frac{f(\gamma)-f(\gamma_X)}{n}.
\]
We remark that, while it is a notation nightmare,
\(c^f(\gamma) = \Phi^f_{\gamma_X}(\gamma)\). \(c^f\) maps into \(\CC \) and some
routine work with \(\nabla\) shows that only that if \(c^f=c^{f'}\) then,
\(\nabla f\) and \(\nabla f'\) are in the same tracial
cohomology class---\ie, \(c^f\) only depends on the class of \(\nabla f \) in
\(H^1_{\tr}(D)\). If we define
\(\phi_g: \pi_1^{\tr}\to \CC , \gamma \mapsto c^f(\gamma)\)---where \(\nabla f\)
is in the tracial cohomology class of \(g\)---we get a
\emph{homomorphism} into \(\CC \), as
\begin{align*}
  c^f(\gamma_1\gamma_2) &= \frac{f(\gamma_1\gamma_2) - f(\gamma_X)}{n} \\
             &= \frac{f(\gamma_1\gamma_2) - f(\gamma_1) + f(\gamma_1) -f(\gamma_X)}{n} \\
             &= \frac{f(\gamma_X\gamma_2) - f(\gamma_X) + f(\gamma_1) -f(\gamma_X)}{n} \\
             &= c^f(\gamma_2) + c^f(\gamma_1).
\end{align*}
Where the penultimate equality uses lemma \ref{lm:technical}.
The fact that \(\phi_g\) is a homomorphism is the first step to characterizing
\(\pi_1^{\tr}(D)\).
\begin{lemma}
  The map
  \begin{align*}
	  \Phi: \prod_{g \in H^1_{\tr}} \pi_1^{\tr}(D) &\longrightarrow \prod_{g \in H^1_{\tr}}\CC  \\
    \prod_{g \in H^1_{\tr}} \gamma &\longmapsto \prod_{g \in H^1_{\tr}} \phi_g(\gamma)
  \end{align*}
  is an injective homomophism.
\end{lemma}

\begin{proof}
  The fact that \(\Phi\) is a homomorphism is immediate, as each of the
  \(\phi_g\) are. For injectivity,
  let \(\alpha,\beta \in \pi_1^{\tr}(D)\) such that \(\prod\phi_g(\alpha)=\prod\phi_g(\beta)\). Seeking to show that
  \(\alpha\) and \(\beta\) are trace equivalent, let \(f\) be a global germ and
  \(\gamma\) essentially take \(X\) to \(Z\). Then, once again using lemma \ref{lm:technical},
  \begin{align*}
    f(\gamma\alpha)-f(\gamma\beta) &= f(\gamma\alpha) - f(\gamma) - \left( f(\gamma\beta)-f(\gamma) \right) \\
                &= c^f(\alpha) - c^f(\beta)
  \end{align*}
  But since \(\prod\phi_g(\alpha)=\prod\phi_g(\beta)\), and \(c^f\) only depends
  on the class of \(\nabla f\),
  \(c^f(\alpha) = c^f(\beta)\). Thus, \(\alpha\) and \(\beta\) are trace
  equivalent and we have shown injectivity.
\end{proof}

Note that lemma also tells us that \(\pi_1^{\tr}(D)\) is both commutative and
torsion free as is injects into a commutative, torsion free group (namely a
product of \(\CC \)'s). With this, we can also show that \(\pi_1^{\tr}(D)\)
needs to be divisible as well. First, note that for any path \(\gamma\),
\[
  \gamma\oplus\gamma_X = \gamma_X\oplus\gamma,
\]
since
\[
  H(t,\theta)  = \begin{bmatrix} \cos \theta & \sin \theta \\ -\sin \theta& \cos \theta \end{bmatrix}
  \left( \gamma \oplus \gamma_X \right)
 \begin{bmatrix} \cos \theta & \sin \theta \\ -\sin \theta& \cos \theta \end{bmatrix}^*
\]
is a homotopy between the paths. Then we see that
\begin{align*}
  \gamma &= \underbrace{\begin{bmatrix} \gamma &&&\\ &\gamma&\\&&\ddots\\&&&\gamma\end{bmatrix}}_{k+1\textrm{-times}}\\
    &=\begin{bmatrix} \gamma &&&\\ &\gamma_X&\\&&\ddots\\&&&\gamma_X\end{bmatrix}
      \begin{bmatrix} \gamma_X &&&\\ &\gamma&\\&&\ddots\\&&&\gamma_X\end{bmatrix}\cdots
      \begin{bmatrix} \gamma_X &&&\\ &\gamma_X&\\&&\ddots\\&&&\gamma\end{bmatrix}\\
    &=\begin{bmatrix} \gamma &&&\\ &\gamma_X&\\&&\ddots\\&&&\gamma_X\end{bmatrix}
      \begin{bmatrix} \gamma &&&\\ &\gamma_X&\\&&\ddots\\&&&\gamma_X\end{bmatrix}\cdots
      \begin{bmatrix} \gamma &&&\\ &\gamma_X&\\&&\ddots\\&&&\gamma_X\end{bmatrix}\\
    &=\begin{bmatrix} \gamma &&&\\ &\gamma_X&\\&&\ddots\\&&&\gamma_X\end{bmatrix}^k,
\end{align*}
and so \(\pi_1^{\tr}(D)\) is divisible. As there is only one way (up to
isomorphism, of course) to be a divisible, torsion free group, we
have completely characterized \(\pi_1^{\tr}(D)\)!

\begin{theorem}
  For \(D\) an anchored free set,
  \[
    \pi_1^{\tr}(D) \simeq \bigoplus_{i \in I} \QQ = \QQ ^I
  \]
  for some set \(I\).
\end{theorem}

\section{Computing the Tracial Fundamental Group}%
\label{sec:interplay}

While this structure theorem is useful, it gets us no closer to actually
\emph{computing} \(\pi_1^{\tr}(D)\) or \(H^1_{\tr}(D)\). Unfortunately, there
is nothing analogous to Van Kampen's theorem or the Mayer Vietoris sequence. For
simple domains, we have some basic tools. Recall the following definion relating
to abelian groups:

\begin{definition}[Rank]%
\label{def:rank}
  For an abelian group, \(G\), the \textbf{rank} of \(G\) is the maximal size of
  a linearly independent subset. That is, it the maximal size of a set
  \(\{g_1, g_2, \dots ,g_k\} \subset G\) such that
  \[
    \sum_{i=0}^k n_ig_i =0 \implies n_i=0 \textrm{ for all }i.
  \]
\end{definition}

\begin{theorem}
  Let \(D\) be a free anchored set. Then,

\begin{enumerate}
  \item \(\dim H_{\tr}^1(D) \leq \rk \pi_1^{\tr}(D)\) whenever both
      quantities are at most countably infinite
  \item \(\dim H_{\tr}^1(D)\neq 0\) if and only if \(\rk \pi_1^{\tr}(D)\neq 0\)
\end{enumerate}
\end{theorem}

\begin{proof}
  \phantom{hello!}
\begin{enumerate}
    \item We can restrict ourselves to the case where \(\rk \pi_1^{\tr}(D)\) is
        finite. Let \(\gamma_1, \dots ,\gamma_k\) be a maximally linearly
        independent set of paths, and suppose that
        \(g_1, \dots , g_{k+1} \in H^1_{\tr}(D)\) is linearly independent. Now
        consider the matrix \([\phi_{g_j}(\gamma_i)]_{ij}\), which is clearly
        singular. If \((\alpha_1, \alpha_2, \dots, \alpha_{k+1})\) is a
        nontrivial vector in its
        kernel, then we can define \(g = \sum_{j=0}^{k+1} \alpha_j g_j\). By
        construction, \(g(\gamma_i)=0\) for all \(i\). Since \(\{\gamma_j\} \)
        is maximal, it follows that \(g\) is the zero function, contradicting
        our assumption that \(\{g_i\} \) is linearly independent.

    \item Suppose that \(\rk \pi_1^{\tr}(D) \neq 0\)---hence there is at least
        one \(\gamma\), a global germ, \(f\), and a path \(\beta\) essentially
        taking \(X\) to \(Z\) such that \(f(\beta) \neq f(\beta\gamma)\). It
        follows that \(f\) does not have a global extension to all of
        \(D\)---hence \(\nabla f\) is nontrivial in \(H^1_{\tr}(D)\).

        Conversely, suppose that \(\rk \pi_1^{\tr}(D)=0\). By part 1 of this
        same theorem,
        \(\dim H^1_{\tr}(D)=0\) as well.
\end{enumerate}
\end{proof}

This bound on the dimension of \(H^1_{\tr}(D)\) is useful, but only if we have some
way to reliably compute \(\pi_1^{\tr}(D)\). Under certain circumstances---which
are not particularly difficult to satisfy---we can compute \(\pi_1^{\tr}(D)\) as
a direct limit of groups by looking at the levelwise homology groups. Classically, the
Hurewicz theorem says that
first homology group of a path connected manifold is isomorphic to the
abelization of the fundamental group. Since we require the domain \(D\) to be
path connected, we can leverage this fact to compute \(\pi_1^{\tr}(D)\).

Let \(D\) be an anchored, free, path connected set such that each \(D_n\) is
nonempty. Choose an anchor \(B \subset D\) such that each \(B_n\) is also
nonempty. If \(X \in B_1\) is our base point, then we have a natural gradation
on \(\pi_1^{\tr}(D)\). Let \(\pi_1^{\tr}(D)_n\) denote the subgroup of paths
contained in \(D_n\). For any \(m \in \NN \), there is a natural inclusion map
\(\pi_1^{\tr}(D)_n \hookrightarrow \pi_1^{\tr}(D)_{mn}\) given by
\(\gamma\mapsto \gamma^{\oplus m}\). Since our base point is in on the
``scalar'' level, we get a sequence of maps
\[
  \pi_1^{\tr}(D)_1 \hookrightarrow
  \pi_1^{\tr}(D)_2 \hookrightarrow
  \pi_1^{\tr}(D)_6 \hookrightarrow \cdots \hookrightarrow
  \pi_1^{\tr}(D)_{n!} \hookrightarrow \cdots
\]
As long as one isn't too fearful of universal properties, it is not difficult to
show that the direct limit of this sequence is isomorphic to \(\pi_1^{\tr}(D)\).
Using this result for computation requires understanding the structure of
\(\pi_1^{\tr}(D)_n\). Since \(\pi_1^{\tr}(D)_n\) only contains paths in \(D_n\),
it is isomorphic to a quotient of some subgroup of \(\pi_1(D_n)\)! Moreover,
since we are restricting ourselves to a fixed level, we can leverage the
Hurewicz theorem. Since \(\pi_1^{\tr}(D)_n\) is abelian is actually a quotient
of \(H_1(D_n)\). Thus, we can realize \(\pi_1^{\tr}(D)\) as a direct limit of
quotients of \(H_1(D_{n!})\)!

\section{Some Examples}%
\label{sec:examples}

The tracial fundamental group is a very new idea, so examples of computation
don't abound. Pascoe's paper provides two examples as exercises. We present a
``topological'' proofs and invite the reader to fill in the details.

\begin{example}
  Let \(D = GL(\CC) = \bigcup_{n \in \NN} GL_n(\CC)\). Consider the case where \(n =1\).
  If we view complex numbers as \(1\times 1\) matrices, then
  \(\{\det z = z \neq 0\} = \CC \setminus \{0\} \). Then
  \(\pi_1^{\tr}(GL)_1\) is a quotient of \(H_1(GL_1(\CC))\simeq \ZZ \).
  Since there are no nontrivial torsion free quotients of \(\ZZ \), it must be the case
  that \(\pi_1^{\tr}(GL)_1\simeq\ZZ \) as well.

  Additionally, we know that there is a natural inclusion
  \(\pi_1^{\tr}(GL)_1 \hookrightarrow \pi_1^{\tr}(GL)_2\), and so
  \(\pi_1^{\tr}(GL)_2\) contains a copy of \(\ZZ \). Moreover, given some
  \(\gamma \in \pi_1^{\tr}(GL)_1\), we have that
  \[
    \begin{bmatrix} \gamma&\\&\gamma_X \end{bmatrix} \in \pi_1^{\tr}(GL)_2.
  \]
  Recall that if we square this element, then we get \(\gamma\)---so
  \(\pi_1^{\tr}(GL)_2\) is isomorphic to the group
  \[
    \ZZ \left[\frac{1}{2}\right].
  \]
  Our next inclusion
  \(\pi_1^{\tr}(GL)_2\hookrightarrow \pi_1^{\tr}(GL)_6\) picks up cube roots for
  the same reason---since
  \[
    \begin{bmatrix} \gamma\\&\gamma_X\\&& \gamma_X  \end{bmatrix} =
    \begin{bmatrix} \gamma\\&\gamma\\&&\gamma_X\\&&&\gamma_X\\&&&&\gamma_X\\&&&&&\gamma_X  \end{bmatrix}.
  \]
  Taking the square and cube roots simultaneously, we also obtain 6th roots.
  \[
    \pi_1^{\tr}(GL)_6 \simeq \ZZ \left[\frac{1}{2}, \frac{1}{3}\right]
  \]
  In the \(n\)-th inclusion, then, we pick up \(n\)-th roots and any other
  factors needed for closure---and so we adjoin \(\frac{1}{n}\) to the
  preceding group.
  The direct limit is, therefore,
  \[
    \pi_1^{\tr}(GL) \simeq \ZZ \left[\frac{1}{2},\frac{1}{3},\frac{1}{4}, \dots\right]\simeq \QQ.
  \]
\end{example}

\begin{example}
  Let \(\Lambda \subset \CC \) be finite and define
  \[
    G_\Lambda = \{ X \in \MM \mid det X - \lambda \neq 0 \textrm{ for all
    } \lambda \in \Lambda\}.
  \]
  We compute \(\pi_1^{\tr}(G_\Lambda)\) just as above. First, see that
  \( \left( G_\Lambda \right) _1 = \CC \setminus \Lambda\), and so
  \(\pi_1^{\tr}(G_\Lambda)_1\simeq H_1((G_\Lambda)_1)\simeq \ZZ^{|\Lambda|}\).
  Inclusion into \(\pi_1^{\tr}(G_\Lambda)_2\) picks up square roots, so
  \[
    \pi_1^{\tr}(G_\Lambda)_2 \simeq \ZZ^{|\Lambda|}\left[ \frac{1}{2} \right].
  \]
  Inclusion into \(\pi_1^{\tr}(G_\Lambda)_6\) picks up cube and 6th roots, and so
  on. Therefore, in the direct limit, we see
  \[
    \pi_1^{\tr}(G_\Lambda) \simeq \QQ ^{|\Lambda|}.
  \]
\end{example}
