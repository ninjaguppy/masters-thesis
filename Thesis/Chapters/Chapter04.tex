\chapter{Monodromy, Global Germs, Algebraic Topology}\label{ch:monodromy}

\section{Temp}%
\label{sec:temp2}

{\color{red} This section needs a better title.}

{\color{blue} Monodromy \(\to \) Global Germs (maybe with sheaf theoretic
  language) \(\to \) alg. Topology.}

{\color{blue} A good thing to end on is the fact that \(\pi_1^{tr}(GL)=\QQ \)
and \(\pi_1(G_\Lambda)= \QQ ^{|\Lambda|}\).}



\section{Classical Monodromy}%
\label{sec:classmono}

In the study of functions of a single complex variable, many of the central
theorems surround the idea of analytic continuation. Given some analytic
function \(f\) on a domain \(\Omega \subset\CC \) and a larger domain
\(\overline{\Omega} \supset \Omega\), we can (with sufficient
``niceness'' conditions) extend \(f\) to an analytic function \(g\) on
\(\overline{\Omega}\). In particular, given some path \(\gamma\) which start in
\(\Omega\) we wish to continue \(f\) \emph{along} \(\gamma\) by recomputing the
power series on overlapping disks with their centers on \(\gamma\).
\begin{figure}[h!]
\centering
  \def\svgwidth{0.8\columnwidth}
  \import{./Chapters/img/part 2/}{monodromycurve.pdf_tex}
\caption{Analytic continuation along a curve}
\label{fig:monocurve}
\end{figure}

Our path \(\gamma\) must avoid any potential poles of \(f\) so that we may
compute the power series, but the uniqueness of such an extension is not
obvious. This is where the aforementioned niceness conditions come into play!
For example, consider the follow setup:

\begin{figure}[h!]
\centering
  \def\svgwidth{0.52\columnwidth}
  \import{./Chapters/img/part 2/}{monodromylog.pdf_tex}
\caption{Two paths in \(\CC \)}
\label{fig:monolog}
\end{figure}

\begin{example}%
\label{ex:logmono}
If we let \(f(x)=\Log x\) be the principle branch of the complex logarithm the
defined on the right half plane, and continue \(f\) along \(\gamma_1\) and
\(\gamma_2\) we get two functions \(f_1\) and \(f_2\) which are analytic at
\(\beta\), but they don't agree! In this case, \(f_1(\beta)\) and \(f_2(\beta)\)
disagree by exactly \(2\pi i\).
\end{example}

The monodromy theorem gives sufficient conditions for the continuation along
two curves to agree:

\begin{theorem}[Monodromy I]%
\label{thm:mon1}
  Let \(\gamma_1, \gamma_2\) be two paths from \(\alpha\) to \(\beta\) and
  \(\Gamma_s\) be a fixed-endpoint homotopy between them. If \(f\) can be
  continued along \(\Gamma_s\) for all \(s \in [0,1]\), then the continuations
  along \(\gamma_1\) and \(\gamma_2\) agree at \(\beta\).
\end{theorem}

In the example above, any homotopy between the two paths must pass through the
origin---where \(\Log x\) fails to be analytic---and hence the two continuations
disagree at \(\beta\).
An equivalent formulation of the monodromy theorem concerns extending a
functions to a larger domain:

\begin{theorem}[Monodromy II]%
\label{thm:mon2}
  Let \(U \subset \CC \) be a disk in \(\CC \) centered at \(z_0\) and
  \(f: U \to \CC \) an analytic function. If \(W\) is
  an open, simply connected set containing \(U\) and \(f\) continues along any
  path \(\gamma \subset W\) starting at \(z_0\), then \(f\) has a unique
  extension to all of \(W\).
\end{theorem}

This second formulation gives another perspective on \(\Log x\). In the example,
\(U\) is a disk around \(\alpha\) that stays in the right half plane and \(W\)
is \(\CC \setminus \{0\} \). While \(\Log x\) continues along any path in
\(\CC  \setminus \{0\} \), the larger domain is \emph{not} simply connected, so
monodromy fails.

In practice, after the initial exposure in a first course in complex variables,
no one computes continuations by hand. {\color{red} This could be a paragraph,
  but is it necessary?}

\section{Free Monodromy}%
\label{sec:freemono}

There is an analogous theorem to \cref{thm:mon1,thm:mon2} in the free settings
initial proven by J.E.\ Pasocoe in \cite{pascoeNoncommutative2020}.
In the classic case, the larger set \(W\) must be simply connected. In the free
setting, however, the theorem is much more powerful.

\begin{theorem}[Free Univeresal Monodromy]%
  \label{thm:freemono}
  If \(f\) is an analytic free function defined on some ball \(B \subset D\),
  for \(D\) an open, connected free set.
  Then \(f\) analytically continues along every path in \(D\) if and only if
  \(f\) has a unique analytic continuation to all of \(D\).
\end{theorem}

\begin{proof}
  [Proof (From \cite{pascoeNoncommutative2020})]
  The fact that a unique extension to all of \(D\) implies that \(f\) has a
  continuation along any \(\gamma\) is immediate.

  {\color{red} honestly I don't quite get the proof yet, so this will come later.}
\end{proof}

In the free case, the ``larger'' set need not be simply connected. Analytic
continuations of free functions, then, cannot be used to detect holes in matrix
domains. It will turn out, however, that the tracial and determinental functions
introduced in \cref{sec:TracGrad} can detect holes and produce an analogue of
the fundamental group!

\section{The Germ of Function}%
\label{sec:germs}

{\color{blue} eww the title. I always like Aluffi's chapter called ``a bit of
  algebraic geometry''---could do ``a bit of sheaf theory'' but I don't want to
  scare the reader.}

As studied in complex analytic and measure theoretic settings, if our space is
structured enough functions are defined by their local behavior. This idea can
be generalized to arbitrary topological spaces by stealing from sheaf theory
{\color{red} fix that wording}.

Let \(X\) be a topological space. For any open set \(U\) we can have \(C(U)\),
the ring of continuous functions \(f: U \to \RR \) (where addition and
multiplication are defined point-wise)
{\color{red} Tracial functions fail to be
  a ring but they *are* a group---should I just change this to a group?}.
Given any \(V \subset U\), notice that a continuous function \(f\) on \(U\), we
can restrict \(f\) to \(V\) and maintain continuity. This gives two maps:

\begin{equation*}
\begin{split}
	V &\lhook\joinrel\longrightarrow U \\
  v &\longmapsto v
\end{split} \qquad \qquad \qquad
\begin{split}
	C(U) &\lhook\joinrel\longrightarrow C(V) \\
  f &\longmapsto f \mid _V
\end{split}
\end{equation*}

Notice that the induced function goes the ``other way.'' This construction is an
example of a sheaf of rings\footnote{To be completely rigorous, a sheaf needs
  additional axioms, but the sheaf of continuous functions is one of the
  prototypical examples so the full defition is not needed in this
  context.}---since \(C(U)\) has a ring structure. We can similarly define sheaves
of abelian groups or sets: to each open set in \(X\) we assign a group (or set)
such that there are analogous restriction maps. For our purposes, these will
always be groups/sets of functions and the restriction maps are the natural ones.

We are interested in the general behavior of continuous functions at some
\(x \in X\). Define \(\mathfrak{C}_x\) to be the set of all functions defined on a
neighborhood of \(x\):
\[
  \mathfrak{C}_x = \{f \in C(U) \mid x \in U \subset X \text{ is open}\}.
\]
By convention, we refer to elements of \(\mathfrak{C}_x\) as a pair, \((f,U)\) of
a continuous function and the open set on which it is defined.
In light of the inclusion maps given above, it obvious that \(\mathfrak{C}_x\)
will have ``duplicate'' elements. Therefore, we define an equivalence relation
on \(\mathfrak{C}_x\) by \((f,U) \sim (g,V) \Leftrightarrow\) there exists
\(W \subset U \cap V\) where \(f \mid _W = g\mid _W\). In a sheaf-theoretic
context, \(\faktor{\mathfrak{C}_x}{\sim} \) is called the \textbf{stalk} at \(x\)
and elements of the stalk are \textbf{germs} at \(x\). If we are dealing with
sheaves of groups or sets, this construction remains unchanged! We can still
define the stalk at given point. While it will not come into play, it is worth
noting that the stalk inherits the algebraic structure of the original
sheaf---\eg for a sheaf of rings (or group), the stalk has a natural ring
(group) structure.

Sheafs of rings/groups/sets of functions arise naturally in many areas of
mathematics. For example, if \(X\) happens to be a smooth manifold, we may
replace \(C(U)\) with \(C^\infty(U)\), the ring of smooth functions into
\(\RR \) and then obtain germs of smooth functions.  Similarly, if \(X\) is a
complex manifold we can construct germs of holomorphic functions.

\begin{example}
{\color{red} Should I use the same number?}

Consider, again, example \ref{ex:logmono}. Our function \(f(x)= \Log x\) has
a germ in \(\Omega\). In particular, both \(f_1\) and \(f_2\) belong to the
equivalence class \([(f,\Omega)]\) as all three functions agree on \(\Omega\).
From this, we {\color{red} see the genesis of the name} germ: germs capture the
\emph{local} behavior of function. Colloquially, this is the ``heart'' of a
function similar to the germ of seed.\footnote{Sheaf theory abounds with
  agrarian nomenclature. {\color{red} This is my first footnote. Should the
    thesis have fun math facts like this in the footnotes or should I do away
    with them?}}
\end{example}

{\color{red} Link to monodromy again?}

As usual, lifting this construction to the free context requires some nuance.
For \(U \subset D\) open, the set of tracial functions on \(U\) (denoted
\(C_{\text{tr}}(U)\)) does not for a ring---it is closed under addition but not
multiplication. Given two tracial functions, \(f,g \in C_{\text{tr}}(U)\), we see
that
\begin{align*}
  (f+g)(X \oplus Y) &= f(X \oplus Y) + g(X \oplus Y) \\
                    &= f(X) + f(Y)+g(X) + g(Y) \\
                    &= (f+g) (X) + (f+g)(Y)
\end{align*}
but,
\begin{align*}
  (fg)(X \oplus Y) &= f(X \oplus Y)g(X \oplus Y) \\
                   &= (f(X) + f(Y))(g(X) + g(Y)) \\
                   &= (fg) (X) + (fg)(Y) + f(X)g(Y)+f(Y)g(X).
\end{align*}

Thankfully, however, the construction remains unchanged if we substitute a ring
of functions for an abelian group of functions (with the identity being
\(f \equiv 0\) and inverses given by simply negating the output). In the case of
determinental and free functions (which play a lesser role in the theory to be
developed) there is not a natural algebraic structure for the corresponding
sheaves, so they are simply sheaves of sets.

\section{The Tracial Fundamental Group}%
\label{sec:trpi1}

While Free Monodromy means that free functions cannot detect the topology of
free sets, the same is not true for a general tracial
function! To do so, however, we will need some definitions. {\color{red} Is this
  too conversational?}

\begin{definition}[Anchored]%
  \label{def:anchored}
Let \(D \subset \MM^{g} \) be a connected, open, free set. If there exists a
nonempty, simply-connected, open, free \(B \subset D\), then we say that \(D\)
is \textbf{anchored}.
\end{definition}

\begin{definition}[Global Germ]
  \label{def:globgerm}
  For \(D\) an open set, and \(B \subset D\) its anchor, we call a tracial
  function \(f:B\to \CC \) a \textbf{global germ} if it analytically continues
  along every path in \(D\) which starts in \(B\).
\end{definition}

Paths in


\clearpage


\begin{figure}[h!]
\centering
  \def\svgwidth{0.9\columnwidth}
  \import{./Chapters/img/part 2/}{essentialpath.pdf_tex}
\caption{A path essentiall taking \(X\) to \(Y\)}
\label{fig:esspath}
\end{figure}
