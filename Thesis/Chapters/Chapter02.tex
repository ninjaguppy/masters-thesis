%%% Local Variables:
%%% mode: latex
%%% TeX-master: "classicthesis"
%%% End:
\chapter{A Second Attempt}\label{ch:SecondAttempt}

In seeking a more general theory, the functional calculus defined last chapter
is insufficient---it would be useful to be able to define \emph{new} functions
instead of simply lifting polynomials to matrix domains. In a move that will
feel familiar to any good student of mathematics, we will trade treat the
set of self-adjoint matrices and polynomial and rational functions on them
as prototypical examples of a more general mathematical
object, the so-called \emph{Matrix Universe}.
After defining this new space and the natural maps in
\cref{sec:MatUniv,sec:TracGrad}, we turn our attention to various topologies
places on matrix universes in \cref{sec:TopManUniv}. While the genesis of free
analysis followed \cref{ch:FirstAttempt} (albeit with the usual bumps in the
road that accompany research) modern free analysis looks much more like this
chapter.

\section{Matrix Universes}%
\label{sec:MatUniv}

Beyond the functional calculus, it becomes useful to construct general functions
on spaces of matrices---to do so, we must make this idea of ``spaces of
matrices'' concrete. The largest such space is the so-called \textbf{Matrix
  Universe}---consisting of \(g\)-tuples of matrices of all sizes:
\[
  \MM^g = \bigcup_{n=1}^\infty (M_n(\CC))^g
\]
By convention, when we consider some
\(X = \left( X_1, \dots , X_g \right) \in \MM^g\), we require that the \(X_i\)
are all the same size. Since \(\MM^g\) is such a large set, we often want to
deal with subsets that still carry some of the implicit structure of \(\MM^g\).
\begin{definition}[Free Set]
  \label{def:FreeSet}
  We say \(D \subset \MM^g\) is a \textbf{free set} (also called an nc set) if it is closed with respect
  to direct sums and unitary conjugation. That it
  \begin{enumerate}
    \item \(X,Y \in D \) means \(X\oplus Y \in D\).
    \item For \(X,U\) like-size matrices with \(U\) unitary and \(X \in D\),
          then \(U X U^* = \left( UX_1U^*, \dots , UX_g U^*  \right) \in D \).
  \end{enumerate}
\end{definition}

For the remainder of this text, \(D\) will denote some free set. Using the
terminology of \cite{pascoeFreeNoncommutativePrincipal2020}, let
\(D_n = D \cap M_n(\CC)^g\) be the level-wise slice of all \(n \times n\)
matrices in \(D\). We say that \(D\) is \textbf{nc-open}\footnote{The topology of \(\MM^g\)
  is still in flux and there is not a canonical topology. See section \ref{sec:TopManUniv} for
  the details } (resp.\ \textbf{connected}, \textbf{simply connected}, \textbf{bounded}) if each \(D_n\) is open
(resp.\ connected, simply connected, bounded). Finally, we say that \(D\) is
\textbf{differentiable} if each \(D_n\) is an open \(C^1\) manifold where the
complex tangent space of every \(X \in D_n\) is all of \(M_n(\CC)^g\).
Given some \(X \in \MM^{g} \), there are three associated sets which capture the
structure of free sets.

\begin{definition}[Similarity Envelope]
  \label{def:semenv}
  Given \(X \in \MM^{g} \), a tuple of \(n \times n\) matrices, the
  \textbf{similarity envelope} of \(X\) is the set
  \[
    \{U^* X U \mid  U \in \mathcal{U}_n\}.
  \]
\end{definition}

\begin{definition}[Fiber]
  \label{def:fiber}
  Given \(X \in \MM^{g} \), a tuple of \(n \times n\) matrices, the
  \textbf{fiber} of \(X\) is the set
  \[
    \{X^{\oplus k} \mid  k \in \NN \}.
  \]
  {\color{red} This is my definition. Is this okay? Can I just define things?}
\end{definition}

\begin{definition}[Envelope]
  \label{def:env}
  Given \(X \in \MM^{g} \), a tuple of \(n \times n\) matrices, the
  \textbf{envelope} of \(X\) is the set
  \[
    \{ U^* X^{\oplus k} U \mid k \in \NN, U \in \mathcal{U}_{kn} \}.
  \]
\end{definition}
Notice that if \(X \in D\), then the entire envelope of \(X\) is automatically
in \(D\) as well! Notice that (as shown in example \ref{ex:multivareval})
polynomials respect the envelope of a matrix in a particularly well-behaved way.
Colloquially, we think of all points in the envelope of \(X\) as ``the
same''---this notion is explored in section \ref{sec:TopManUniv} and throughout
chapter \ref{ch:monodromy}.

In the context of \cref{sec:functionalcalc,sec:ExtMuliVarFun}
%
, the domains in the functional calculus were
\(\HH^g = \bigcup_{n=1}^{\infty} {\HH_n}^g\). \(\HH ^g\) is a differentiable a
free.

On \(\MM^{g} \), we define a product that resembles the inner product on
\(\CC^n\). Given \(A,B \in \MM^{g} \) which are \(g\)-tuples of \(n \times n\)
matrices:
\begin{align*}
	\cdot : \MM^{g} \times \MM^{g}  &\longrightarrow M_n(\CC) \\
  \cdot (A,B) = A \cdot B &\longmapsto \sum_{i=1}^g A_i B_i
\end{align*}

{\color{red} James uses this product, but like what in the world is going on
  with it???}

{\color{red} \(tr (A \otimes Id)\) But its more complicated than that bc A is a
``row vector'' of sorts}


\section{Tracial Functions and Uniqueness of the Gradient}%
\label{sec:TracGrad}
Now that we have \(\MM^d\), we can work with general functions on our matrix
universe. As a whole, free analysis is concerned with so-called \emph{free
  functions}, which respect the direct sums and unitary conjugation.
{\color{red} Do they need to be graded?}
\begin{definition}[Free Function]
\label{def:FreeFun}
  A function \(f: D\to \MM^{\color{red} \text{something}}\) is called \textbf{free} if
  \begin{enumerate}
    \item \(f(X\oplus Y)= f(X) \oplus f(Y)\)
    \item \(f(U X U^*) = f(U)f(X)f(U^*)\) where \(X\) and \(U\) are like-size
          and \(U\) is unitary.
  \end{enumerate}
\end{definition}

The two other classes of functions we are concerned with are those that act like
the trace and the determinant:
\begin{definition}[Determinantal Free Function]
  \label{def:DetFreeFun}
  A function \(f: D \to \CC \) is a \textbf{determinantal free function} if
  \begin{enumerate}
    \item \(f(X\oplus Y) = f(X)f(Y)\)
    \item \(f(U X U^*) = f(X)\) where \(X\) and \(U\) are like-size
          and \(U\) is unitary.
  \end{enumerate}
\end{definition}

\begin{definition}[Tracial Free Function]
  \label{def:TrFreeFun}
  A function \(f: D \to \CC \) is a \textbf{tracial free function} if
  \begin{enumerate}
    \item \(f(X\oplus Y) = f(X)+f(Y)\)
    \item \(f(U X U^*) = f(X)\) where \(X\) and \(U\) are like-size
          and \(U\) is unitary.
  \end{enumerate}
\end{definition}

Given a free function of any type, we can define the directional derivative
(Definition \ref{def:DirDeriv}) identically. It is worth noting that, while they
share the moniker of \emph{free}, determinantal and tracial functions are
\emph{not} free functions.
It is only these tracial functions which inherit the gradient mentioned above.
Similarly to traditional multivariable calculus we define the gradient via its
relationship to the directional derivative:
\begin{definition}[Free Gradient]
\label{def:FreeGrad}
  Given a tracial free function \(f\), the free gradient, \(\nabla f\), is the
  unique free function satisfying
  \[
    \tr \left( H \cdot \nabla f(X) \right) = \tr\, Df(X)[H]
  \]
\end{definition}

It is not-at-all obvious that such a \(\nabla f \) should be unique---after all
any linear combination of commutator is has trace zero. {\color{red} should I
  explain this?} In the case that \(f\) is a single-variable function we can
replace \(\nabla f\) with the traditional derivative, \(f'\), as seen in
\cite[Thm 3.3]{pascoeTrace2020}.
\begin{theorem}
  Let \(f: (a,b)\to \RR \) be a \(C^1\) function. Then
  \[
    \tr \, Df(X)[H] = \tr \left( Hf'(X) \right)
  \]
\end{theorem}

The proof in \cite{pascoeTrace2020} simply asserts the uniqueness of a function
\(g(X)\) and then shows that \(g(x)=f'(x)\) for \(x \in (a,b)\). Instead, we can
construct such a \(g\) and recover the theorem along the way:
\begin{proof}

We start with a construction from Bhatia's Matrix Analysis: Let
$f \in C ^{1} (I)$ and define $f ^{[1]} $ on $I \times I$ by
\[
  f^{[1]} (\lambda,\mu) =
  \begin{cases}
    \frac{f(\lambda) - f(\mu)}{\lambda-\mu} & \lambda \neq \mu \\
    f'(\lambda) & \lambda = \mu.
  \end{cases}
\]
We call $f ^{[1]} (\lambda,\mu)$ the \emph{first divided difference} of $f$ at
$(\lambda,\mu)$. If $\Lambda$ is a diagonal matrix with entries
$\{ \lambda_{i}\} $, We may extend $f$ to accept $\Lambda$ by
defining the $(i,j)$-entry of $f ^{[1]} (\Lambda)$ to be
$f ^{[1]} (\lambda_i,\lambda_j)$. If $A$ is a self adjoint matrix with
$A = U \Lambda U ^{*} $, then we define
$f ^{[1]} (A) = U f ^{[1]} (\Lambda) U ^{*} $. Now we borrow a theorem from
Bhatia \cite{bhatiaMatrixAnalysis1997}:
\begin{theorem}[Bhatia V.3.3]

{\color{red} Theorem numbering?}
  Let $f \in C ^{1} (I)$ and let $A$ be a self adjoint matrix with all
  eigenvalues in $I$. Then \[
    Df(A)[H] = f ^{[1]} (A) \circ H,
  \]
  where $\circ$ denotes the Schur-product\footnote{Entrywise} in a basis where $A$ is diagonal.
\end{theorem}

That is, if $A = U   \Lambda U ^{*} $, then
\[
  Df(A)[H] = U \left( f ^{[1]} (\Lambda) \circ (U ^{* } H U) \right)U ^{*}.
\]
%
To prove our claim, we need to take the trace of both sides. Since trace is
invariant under a change of basis, it is clear that
\[
  \tr Df(A)[H] = \tr \left( f ^{[1]} (\Lambda) \circ (U ^{* } H U) \right).
\]
If $U = u_{ij}$, $U ^{*} = \overline{u}_{ij}$ and $H = h_{ij}$, then the
$(i,j)$-entry of $U ^{*}HU$ is
\[
  {(U ^{* } H U)}_{ij} = \overline{u}_{ik}h_{k\ell}u_{\ell j}
\]
Where we sum over the duplicate indices $k$ and $\ell$. While the structure of
$f ^{[1]} (\Lambda)$ is a bit unruly, our diagonal entries are $f'(\lambda)$.
This means that when we take the trace of the Schur product, we have
\[
 \sum_k\sum_\ell \sum_i f'(\lambda_i)\overline{u}_{ik}h_{k\ell}u_{\ell i}
\]
Now consider the matrix product
$U\, \text{diag} \{f'(\lambda_1), \dots ,f'(\lambda_n)\} \,U ^{*} H $. Since one of our terms
is diagonal, the trace of this multiplication is simple:
\[
  \text{tr}\; U \,\text{diag} \{f'(\lambda_1), \dots ,f'(\lambda_n)\}\, U ^{*} H
  = \sum_k\sum_\ell\sum_i  u_{ik}f'(\lambda_k) \overline{u}_{k \ell} h_{\ell i}
\]
Since \(u_{ik}, \overline{u}_{k\ell}, h_{\ell i} \in \CC \) they commute. We can
then relabel our indices
$i \mapsto \ell\; \ell \mapsto k \; k \mapsto i $ to get
\[
  \tr\; U \,\text{diag} \{f'(\lambda_1), \dots ,f'(\lambda_n)\}\, U ^{*} H
  = \sum_k\sum_\ell \sum_i f'(\lambda_i) \overline{u}_{i k} h_{k \ell}u_{\ell i},
\]
So, for every direction \(H\), we have that
$\tr \left( U\, \text{diag} \{f'(\lambda_1), \dots ,f'(\lambda_n)\} \,U ^{*} H\right) =
   \tr \left( f ^{[1]} (\Lambda) \circ (U ^{* } H U) \right). $
{\color{red} overfull hbox :eyeroll:}
By picking the ``correct'' \(H\),
\footnote{See the proof of \ref{thm:trdual} for the details of how to pick the
  \(H\)'s}
we conclude that our unique quantity \(g(X)\) is
\(U\, \text{diag} \{f'(\lambda_1), \dots ,f'(\lambda_n)\} \,U ^{*} \). But,
recall that \(X=U\Lambda U\) so, in the functional calculus, $g(X) = f'(X)$.
This recovers theorem 3.3 of \cite{pascoeTrace2020} as we have constructed a
\(g\) such that
\[
  \tr \; Df(X)[H] = \tr H g(X)
\]
\end{proof}

With our theorem proven, we turn our attention back to the \(\nabla f\). The
single variable case motivates that \(\nabla f\) should correspond to the
standard gradient from vector calculus. With some work, the above proof lifts
the multi-variable case. It will be instructive, however, to consider a
different proof.

\begin{theorem}[Trace Duality]%
\label{thm:trdual}
Let \(f,g\) be free functions \(\MM^{g} \to \MM^{\tilde{g}} \). If
\(\tr H \cdot f = \tr H \cdot g\) for all tuples \(H\), then \(f=g\).
\end{theorem}

\begin{proof}
  Since the trace relation holds for all $H$, we may choose our $H$ carefully to
  show the equality of $f$ and $g$. Say that $H,f(X),g(X)$ are $g$-tuples of
  matrices---we will first show that $f_1=g_1$ and we will do so entry by entry.
  Let $E_{ij}$ be the matrix will all zeroes and a 1 in the $(i,j)$-entry.  Now
  let $H= (E_{ji},0, \dots ,0)$. So $\tr E_{ji}f_1(X) = \tr E_{ji}g_1(X)$.
  In our products, the only elements on the diagonal are $(f_1(X))_{ij}$ and
  $(g_1(X))_{ij}$, so when we take the trace we have $(f_1(X))_{ij} =(g_1(X))_{ij}$. If we
  do this for every $(i,j)$, we see that $f_1(X)=g_1(X)$. Similarly, we can choose
  \(H = ( 0, E_{ji},0, \dots, 0)\) for each \(i,j\) to show that \(f_2(X)=g_2(X)\) and
  so on. Since \(f(X)=g(X)\) for each \(X \in \MM^{g} \), it follows that \(f=g\).
\end{proof}

Admittedly, there is a slight complication that is overlooked in the above proof
when it comes to the domains of \(f\) and \(g\). Where these domains overlap, we
can consider them as the same function (and therefore \(\nabla f \) is unique)
but if \(f\) is defined on \(D\) and \(g\) is defined on \(\tilde{D}\), then the
above proof only holds on \(D \cap \tilde{D}\). Examples of such \(f\) and \(g\)
abound when considering rational functions, which are explored in
\cref{sec:ncrational}.

\section{The Topology of Matrix Universes }%
\label{sec:TopManUniv}

At the time of writing, there is no ``canonical topology'' for \(\MM^g\). For a
long time it seemed like the \emph{free} topology (to be defined below) was the
obvious choice, but recent work (c.f.\ \cite{pascoeentire2019}) has shown that the free
topology does not put enough structure on \(\MM^g\). See \cite{aglerAspects2016}
for a full treatment of the common topologies on \(\MM^g\).

A naive approach to a topology on \(\MM = \bigcup_n M_n(\CC)\) would be the
disjoint union topology---which is then extended do a topology on \(\MM^g\) via
the product topology. Notice, however that this ignores a significant amount of
the implicit structure of nc-sets as we get a disconnected space with countable
many connected components. Topologically, this is means that means that
\[
  H_\bullet (D) = \bigoplus_{n \in \NN } H_\bullet(D_n).
\]
At first glance, this seems fine enough, but it ignores the fact that for
\(X \in D\) we require \(X^{\oplus k} \in D\) for all \(k\) and \(U^*XU \in D\)
for all unitary \(U\). In a sense, we think of the all the direct sums of \(X\)
and its similarity envelope as ``the same.'' In light of this, if \(\sim\) is
the equivalence relation that \(X\sim Y\) if \(Y = X^{\oplus k}\) or
\(Y = U^*XU\) {\color{red} is this actually an equivalence relation? The second
statement is immediate but the first isn't an eq.\ rel.},
then any useful topological theory on \(D \subset \MM^g\) should descend to
classic theory on \(\faktor{D}{\sim}\). One needs only look at \(H_0(D)\) to see
that the naive approach fails to give useful information. It should be the case
that \(H_0(\MM^{g} )\) is trivial but in the disjoint union topology it is easy
to see
\[
  H_0(\MM^{g} ) = \bigoplus_{n \in \NN } \ZZ ,
\]
which does not behave as we would expect.

{\color{blue} a note on convergence somewhere.}

\subsection{Admissible Topologies}%
\label{sec:admtopo}

{\color{blue} a cool example is showing that \(\HH \) is dense in \(\MM\)}

In light of the above discussion, we will present some of the candidate
topologies which show some promise in understanding the topology on \(\MM^g\)
and its subsets.
We say that a topology \(\tau\) is \textbf{admissible} if it has a basis of nc
bounded open sets, \(D\) (recall that this means that \(D\) is closed under
direct sums and unitary conjugation, and that each \(D_n\) is a bounded open set
in \(M_n(\CC)^g\)). The finest admissible topology is the so-called \textbf{fine
  topology},
the basis of which consists of \emph{all} nc open sets.

A slightly more restrictive topology (that seems to show some promise in the eyes
of the author) is the \textbf{fat} topology. For \(n \in \NN \),
\(r \in \RR ^+\), and \(X \in \MM^g_n\), we first define a matricial polydisc
\[
  D_n(X,r) := \{A \in \MM^g \mid \max_{1\leq i\leq g} \| X_i-A_i \| <r\}.
\]
Now we sweep \(D_n\) through all direct sum copies of \(X\):
\[
  D(X,r) := \bigcup_{k=1}^\infty D_{kn} (X^{\oplus k},r)
\]
Finally, we take the similarity envelope of \(D(X,r)\)
\[
  F(X,r) :=  \bigcup_{n=1}^\infty \bigcup_{U \in \mathcal{U}_n} U^* \left( D(X,r) \cap \MM^g_n \right) U
\]
Both the fine and the fat topologies admit implicit function theorems.

The final candidate topology is the aforementioned \textbf{free} topology.
Recall that \(\RR \langle x \rangle \) is the algebra of nc polynomials over the
real number and that
\(\RR \langle x \rangle ^{k\times k}\) is the set of \(k \times k\) matrices
with entries in \(\RR \langle x \rangle \). Let
\(\delta \in \RR \langle x \rangle ^{k\times k}\) and define
\[
  G_\delta = \{x \in \MM^{g} \mid \| \delta(x) \| <1\}
\]
The set of all \(G_\delta\) as \(k\) ranges over \(\ZZ ^+\) form the basis for
the free topology. Indeed, any \(X \in \MM^{g} \) is trivially in one of the
\(G_\delta\) (take \(\delta=X\)) and with some work one can show that
\(G_{\delta_1} \cap G_{\delta_2}= G_{\delta_1\oplus \delta_2}\) ({\color{red}
  prove this}) so we do,
indeed, have a basis.

\section{Free Analogues of Classical Results}%
\label{sec:freeanal}



In \cite{aglerGlobal2013}, Agler and McCarthy proved an free analogue of the
Oka-Weil theorem: any holomorphic function on a compact set in the free topology
can be uniformly approximated by polynomials. Unfortunately, it was later proven
in \cite{pascoeentire2019} and \cite{augatCompact2017} that the only compact
sets in the \(\MM^{g} \) are the envelope of finitely many points, trivializing
the result of Agler and McCarthy.

For the rest of this thesis, we will be using the conventions mentioned in
\cref{sec:MatUniv}: \( D \subset\MM^{g} \) open if each \(D_n\) is open---these
are precisely the basic open sets in the fine topology.
Given \(X,Y \in D\), it is not generally true that we can separate \(X\) and
\(Y\) with open sets. However if \(Y\) is not in the similarity envelope of
\(X\) and \(X\) and \(Y\) have disjoint fibers, then we \emph{can} separate
them! Motivated by definitions in section \ref{sec:trpi1} we call a topology
satisfying this condition (Hausdorff outside of the similarity envelope and
fiber) \textbf{essentially Hausdorff}.

\section{nc Rational Functions}%
\label{sec:ncrational}

{\color{blue} Short review about defining rational functions via equivalence
  classes. We need this bc rational functions give the nc picard group and
  divisors and all that }
