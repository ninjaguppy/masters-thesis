\documentclass[12pt]{exam}

% opening
\pagestyle{headandfoot}
\runningheadrule
\runningheader{Lucas Kerbs}
{Trace Duality, Page \thepage\ of \numpages}
{\today}
\runningfooter{}{}{}

% load packages
\usepackage{amsmath,amsfonts,amssymb,amsthm, mathtools, mathrsfs, faktor, bbm}
% \usepackage{pgfplots}
\usepackage{wrapfig}
\usepackage{array}
% \usepackage{gensymb}
\usepackage{graphicx}
\usepackage{tikz}
\usepackage{tikz-cd}
\usepackage{enumerate}
\usepackage[margin=1.0in]{geometry}
\newtheorem{theorem}{Theorem}

\usepackage{import}
\usepackage{xifthen}
\usepackage{pdfpages}
\usepackage{transparent}

\newcommand{\incfig}[1]{%
  \def\svgwidth{0.5\columnwidth}
  \import{./images/}{#1.pdf_tex}
}

% General shortcuts
\newcommand{\RR}{\mathbb{R}}
\newcommand{\CC}{\mathbb{C}}
\newcommand{\ZZ}{\mathbb{Z}}
\newcommand{\QQ}{\mathbb{Q}}
\newcommand{\NN}{\mathbb{N}}
\newcommand{\Id}{\mathbbm{1}}
\newcommand{\wms}{\textsc{wms} }
\newcommand{\wma}{\textsc{wma} }
\newcommand{\wwlog}{\textsc{wlog} }
\newcommand{\sol}{\paragraph{Solution:}}
\newcommand{\ssol}{\subparagraph{Solution:}}
\renewcommand\qedsymbol{$\blacksquare$}

\begin{document}

\begin{center}
  {\Large Trace Duality}
\end{center}

\emph{Claim:} If $f \in C ^{1} (I)$, and $X,H$ are self adjoint matrices, then
there is a unique quantity $g(X)$ such that
\[
  \text{tr} Df(X)[H] = \text{tr} Hg(X).
\]

\bigskip

We start with a construction from Bhatia's Matrix Analysis: Let
$f \in C ^{1} (I)$ and define $f ^{[1]} $ on $I \times I$ by
\[
  f (\lambda,\mu) =
  \begin{cases}
    \frac{f(\lambda) - f(\mu)}{\lambda-\mu} & \lambda \neq \mu \\
    f'(\lambda) & \lambda = \mu.
  \end{cases}
\]
We call $f ^{[1]} (\lambda,\mu)$ the \emph{first divided difference} of $f$ at
$(\lambda,\mu)$. If $\Lambda$ is a diagonal matrix with entries
$\{ \lambda_{i}\} $, We may extend $f$ to accept $\Lambda$ by
defining the $(i,j)$-entry of $f ^{[1]} (\Lambda)$ to be
$f ^{[1]} (\lambda_i,\lambda_j)$. If $A$ is a self adjoint matrix with
$A = U \Lambda U ^{*} $, then we define
$f ^{[1]} (A) = U f ^{[1]} (\Lambda) U ^{*} $. Now we borrow a theorem from Bhatia:
\begin{theorem}[Bhatia V.3.3]
  Let $f \in C ^{1} (I)$ and let $A$ be a self adjoint matrix with all
  eigenvalues in $I$. Then \[
    Df(A)[H] = f ^{[1]} (A) \circ H,
  \]
  where $\circ$ denotes the Schur-product in a basis where $A$ is diagonal.
\end{theorem}

That is, if $A = U   \Lambda U ^{*} $, then
\[
  Df(A)[H] = U \left( f ^{[1]} (\Lambda) \circ (U ^{* } H U) \right)U ^{*}.
\]
%
To prove our claim, we need to take the trace of both sides. Since trace is
invariant under a change of basis, it is clear that
\[
  \text{tr}Df(A)[H] = \text{tr} \left( f ^{[1]} (\Lambda) \circ (U ^{* } H U) \right).
\]
If $U = u_{ij}$, $U ^{*} = \overline{u}_{ij}$ and $H = h_{ij}$, then the
$(i,j)$-entry of $U ^{*}$ is
\[
  {(U ^{* } H U)}_{ij} = \overline{u}_{ik}h_{k\ell}u_{\ell j}
\]
Where we sum over the duplicate indices $k$ and $\ell$. While the structure of
$f ^{[1]} (\Lambda)$ is a bit unruly, our diagonal entries are $f'(\lambda)$.
This means that when we take the trace of the Schur product, we have
\[
  \sum_i f(\lambda_i)\overline{u}_{ik}h_{k\ell}u_{\ell i}
\]
Now consider the matrix product
$U\, \text{diag} \{f'(\lambda_1), \dots ,f'(\lambda_n)\} \,U ^{*} H $. Since one of our terms
is diagonal, the trace of this multiplication is simple:
\[
  \text{tr}\; U \,\text{diag} \{f'(\lambda_1), \dots ,f'(\lambda_n)\}\, U ^{*} H
  = \sum_k u_{ik}f(\lambda_k) \overline{u}_{k \ell} h_{\ell i}
\]
Since our entries commute, we can relable our indices
$i \mapsto \ell\; \ell \mapsto k \; k \mapsto i $ to get
\[
  \text{tr}\; U \,\text{diag} \{f'(\lambda_1), \dots ,f'(\lambda_n)\}\, U ^{*} H
  = \sum_i f(\lambda_i) \overline{u}_{i k} h_{k \ell}u_{\ell i},
\]
See that the quantity a
$U\, \text{diag} \{f'(\lambda_1), \dots ,f'(\lambda_n)\} \,U ^{*} H $ is
independent of our choice of $H$, so is it the needed quantity $g(X)$. Further,
since $X = U \Lambda U ^{*} $, $g(X) = f'(X)$. This recovers theorem 3.3 of
\emph{LEARN TO DO CITATIONS} as we have
\[
  \text{tr} \; Df(X)[H] = \text{tr} H g(X)
\]

\bigskip

\bigskip

\bigskip



\noindent Now we turn our attention to the example in \emph{citation}. I verified that
\begin{align*}
  \text{tr} H \cdot \text{div} \left( 1+XY \right)
    &= \text{tr} \left( H_1 Y+X H_2 \right)(1+XY) ^{-1}\\
    &= \text{tr} H_1Y(1+XY) ^{-1} + H_2(1+XY) ^{-1} X \\
    &= \text{tr} \left( H_1,H_2 \right) \cdot \left( Y(1+XY) ^{-1} , \left( 1+XY \right) ^{-1}X \right),
\end{align*}
but how can we be confident that this means we have
$\text{div} \left( 1+XY \right) = ( Y(1+XY) ^{-1} ,$ $ \left( 1+XY \right) ^{-1}X)$?
We consider the general case: Say that
$\text{tr} H \cdot f = \text{tr} H \cdot g$. Since this holds for all $H$, we
may choose our $H$ carefully to show the equality of $f$ and $g$. Say that $H,f,g$
are $k$-tuples---we will first show that $f_1=g_1$ and we will do so entry by
entry. Let $E_{ij}$ be the matrix will all zeroes and a 1 in the $(i,j)$-entry.
Now let $H= (E_{ji},0, \dots ,0)$. So $\text{tr } E_{ji}f_1 = \text{tr
} E_{ji}g_0$. In our products, the only elements on the diagonal are
$(f_1)_{ij}$ and $(g_1)_{ij}$, so when we take the trace we have
$(f_1)_{ij} =(g_1)_{ij}$. If we do this for every $(i,j)$, we see that
$f_1=g_1$. Showing that the other components are equal is identical.

\emph{\bf QQ:} Is this necessarily true? This is very nitpicky, but does the fact
that $f$ and $g$ have the same components mean they have the same expression? I
see how it would make the same function, but I see a potential issue with
domain---if they have different domains then they could have different
expressions but still have the same entries when the domains overlap.


\end{document}
