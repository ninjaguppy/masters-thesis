\documentclass[11pt]{exam}

% opening
\pagestyle{headandfoot}
\runningheadrule
\runningheader{Lucas Kerbs}
{Lucas Kerbs Thesis Proposal, Page \thepage\ of \numpages}
{\today}
\runningfooter{}{}{}

% load packages
\usepackage{amsmath,amsfonts,amssymb,amsthm, mathtools, mathrsfs, faktor, bbm}
% \usepackage{pgfplots}
\usepackage{wrapfig}
\usepackage{array}
% \usepackage{gensymb}
\usepackage{graphicx}
\usepackage{tikz}
\usepackage{enumerate}
\usepackage[margin=1.250in]{geometry}

\usepackage{import}
\usepackage{xifthen}
\usepackage{pdfpages}
\usepackage{transparent}

\newcommand{\incfig}[1]{%
  \def\svgwidth{0.5\columnwidth}
  \import{./images/}{#1.pdf_tex}
}

% General shortcuts
\newcommand{\RR}{\mathbb{R}}
\newcommand{\CC}{\mathbb{C}}
\newcommand{\ZZ}{\mathbb{Z}}
\newcommand{\QQ}{\mathbb{Q}}
\newcommand{\NN}{\mathbb{N}}
\newcommand{\Id}{\mathbbm{1}}
\newcommand{\wms}{\textsc{wms} }
\newcommand{\wma}{\textsc{wma} }
\newcommand{\wwlog}{\textsc{wlog} }
\newcommand{\sol}{\paragraph{Solution:}}
\newcommand{\ssol}{\subparagraph{Solution:}}
\renewcommand\qedsymbol{$\blacksquare$}

\begin{document}
\begin{center} {\LARGE Thesis Proposal \vspace{2mm}}\\ Lucas Kerbs \\ \today \end{center}

Noncommutative function theory is a relatively new branch of mathematics which
combines analysis, linear and abstract algebra, and topology. It
seeks to understand functions (usually polynomials and rational functions) with
noncommuting variables and treats them as both algebraic objects and transformations
of matrices of arbitrary size. A recent paper by J.E. Pascoe explored the
topology of the domains of these functions via universal monodromy and the
construction of so-called a `tracial' fundamental group and co-homology.

The theory of noncommutative functions grew out of the study of several complex
variables. It is not a large leap of abstraction from considering functions of
several complex variables---evaluated on \(\CC ^n\)---to studying functions
which are instead evaluated on tuples of square matrices,
\(\left(\CC ^{n\times n}\right)^g\). For these new functions, we notably lose the
commutativity of the variables; in light of this, we often call them \emph{nc}
functions or \emph{free} functions. The domains of theses functions can be
notoriously tricky---we require that it is closed under direct sums of its
elements as well as conjugation by unitary matrices. While this is not
particularly restrictive in the case of an nc polynomial, it is much more
complex if we consider nc rational functions or arbitrary free domains.

In the study of complex variables, the universal monodromy theorem tells us that
given a complex domain \(U\) and an analytic \(f: U \to \CC \), if \(f\) can be
analytically continue along any path in \(W\supset U\), then we have an analytic
\(g:W\to \CC \) such that \(\left. g\right|_{U}=f\). The beauty of Pascoe's
recent paper leverages the fact that nc domains are complex manifolds and thus
we can use universal monodromy in order to study the \emph{topological}
structures of these sets. Monodromy allows for the construction of well-defined
fundamental and co-homology groups (despite the convoluted structure of the
domains). Notably, the fundamental group of these domains is always abelian.


The goal of my thesis will be three-fold. First will be an expository exercise
on Pascoe's paper; the goal will be to explicate the paper and fill in the gaps
in some of his proofs. Secondly is the calculation of these groups for the
domains of common nc rational functions. While the paper computes these
structures for a small handful of domains (for example
\(\pi_1^{tr}(GL)\simeq \QQ \), where \(GL\) is the set of complex valued
matrices with nonzero determinant), there has been very little effort to perform
actual calculations beyond the few done in the initial paper. Pascoe's paper ends
with a series of open questions and conjectures---the final goal of my thesis is
to explore these questions.  While being able to answer one of them fully is
likely outside the scope of possibility, the larger goal is to unpack the
questions and their meaning in the hopes of making meaningful progress in the
direction of an answer.


\end{document}
